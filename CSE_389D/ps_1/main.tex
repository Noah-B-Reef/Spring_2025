\documentclass[12pt]{report}
\usepackage{scribe,graphicx,graphics}

\usepackage{siunitx}
\course{CSE 389D} 	
\coursetitle{Mathematical Modeling}	
\semester{Spring 2025}
\lecturer{} % Due Date: {\bf Mon, Oct 3 2016}}
\lecturetitle{Problem Set}
\lecturenumber{1}   
\lecturedate{}    
\usepackage{enumerate}
\newcommand{\remind}[1]{\textcolor{red}{\textbf{#1}}} %To remind me of unfinished work to fix later
\newcommand{\hide}[1]{} %To hide large blocks of code without using % symbols

\newcommand{\ep}{\varepsilon}
\newcommand{\vp}{\varphi}
\newcommand{\lam}{\lambda}
\newcommand{\Lam}{\Lambda}
%\newcommand{\abs}[1]{\ensuremath{\left\lvert#1\right\rvert}} % This clashes with the physics package
%\newcommand{\norm}[1]{\ensuremath{\left\lVert#1\right\rVert}} % This clashes with the physics package
\newcommand{\floor}[1]{\ensuremath{\left\lfloor#1\right\rfloor}}
\newcommand{\ceil}[1]{\ensuremath{\left\lceil#1\right\rceil}}
\newcommand{\A}{\mathbb{A}}
\newcommand{\B}{\mathbb{B}}
\newcommand{\C}{\mathbb{C}}
\newcommand{\D}{\mathbb{D}}
\newcommand{\E}{\mathbb{E}}
\newcommand{\F}{\mathbb{F}}
\newcommand{\K}{\mathbb{K}}
\newcommand{\N}{\mathbb{N}}
\newcommand{\Q}{\mathbb{Q}}
\newcommand{\R}{\mathbb{R}}
\newcommand{\T}{\mathbb{T}}
\newcommand{\X}{\mathbb{X}}
\newcommand{\Y}{\mathbb{Y}}
\newcommand{\Z}{\mathbb{Z}}
\newcommand{\As}{\mathcal{A}}
\newcommand{\Bs}{\mathcal{B}}
\newcommand{\Cs}{\mathcal{C}}
\newcommand{\Ds}{\mathcal{D}}
\newcommand{\Es}{\mathcal{E}}
\newcommand{\Fs}{\mathcal{F}}
\newcommand{\Gs}{\mathcal{G}}
\newcommand{\Hs}{\mathcal{H}}
\newcommand{\Is}{\mathcal{I}}
\newcommand{\Js}{\mathcal{J}}
\newcommand{\Ks}{\mathcal{K}}
\newcommand{\Ls}{\mathcal{L}}
\newcommand{\Ms}{\mathcal{M}}
\newcommand{\Ns}{\mathcal{N}}
\newcommand{\Os}{\mathcal{O}}
\newcommand{\Ps}{\mathcal{P}}
\newcommand{\Qs}{\mathcal{Q}}
\newcommand{\Rs}{\mathcal{R}}
\newcommand{\Ss}{\mathcal{S}}
\newcommand{\Ts}{\mathcal{T}}
\newcommand{\Us}{\mathcal{U}}
\newcommand{\Vs}{\mathcal{V}}
\newcommand{\Ws}{\mathcal{W}}
\newcommand{\Xs}{\mathcal{X}}
\newcommand{\Ys}{\mathcal{Y}}
\newcommand{\Zs}{\mathcal{Z}}
\newcommand{\ab}{\textbf{a}}
\newcommand{\bb}{\textbf{b}}
\newcommand{\cb}{\textbf{c}}
\newcommand{\db}{\textbf{d}}
\newcommand{\ub}{\textbf{u}}
\newcommand{\sbb}{\textbf{s}}
%\renewcommand{\vb}{\textbf{v}} % This clashes with the physics package (the physics package already defines the \vb command)
\newcommand{\wb}{\textbf{w}}
\newcommand{\xb}{\textbf{x}}
\newcommand{\yb}{\textbf{y}}
\newcommand{\zb}{\textbf{z}}
\newcommand{\vbb}{\textbf{v}}
\newcommand{\Ab}{\textbf{A}}
\newcommand{\Bb}{\textbf{B}}
\newcommand{\Cb}{\textbf{C}}
\newcommand{\Db}{\textbf{D}}
\newcommand{\eb}{\textbf{e}}
\newcommand{\ex}{\textbf{e}_x}
\newcommand{\ey}{\textbf{e}_y}
\newcommand{\ez}{\textbf{e}_z}
\newcommand{\zerob}{\mathbf{0}}
\newcommand{\abar}{\overline{a}}
\newcommand{\bbar}{\overline{b}}
\newcommand{\cbar}{\overline{c}}
\newcommand{\dbar}{\overline{d}}
\newcommand{\ubar}{\overline{u}}
\newcommand{\vbar}{\overline{v}}
\newcommand{\wbar}{\overline{w}}
\newcommand{\xbar}{\overline{x}}
\newcommand{\ybar}{\overline{y}}
\newcommand{\zbar}{\overline{z}}
\newcommand{\Abar}{\overline{A}}
\newcommand{\Bbar}{\overline{B}}
\newcommand{\Cbar}{\overline{C}}
\newcommand{\Dbar}{\overline{D}}
\newcommand{\Ubar}{\overline{U}}
\newcommand{\Vbar}{\overline{V}}
\newcommand{\Wbar}{\overline{W}}
\newcommand{\Xbar}{\overline{X}}
\newcommand{\Ybar}{\overline{Y}}
\newcommand{\Zbar}{\overline{Z}}
\newcommand{\Aint}{A^\circ}
\newcommand{\Bint}{B^\circ}
\newcommand{\limk}{\lim_{k\to\infty}}
\newcommand{\limm}{\lim_{m\to\infty}}
\newcommand{\limn}{\lim_{n\to\infty}}
\newcommand{\limx}[1][a]{\lim_{x\to#1}}
\newcommand{\liminfm}{\liminf_{m\to\infty}}
\newcommand{\limsupm}{\limsup_{m\to\infty}}
\newcommand{\liminfn}{\liminf_{n\to\infty}}
\newcommand{\limsupn}{\limsup_{n\to\infty}}
\newcommand{\sumkn}{\sum_{k=1}^n}
\newcommand{\sumk}[1][1]{\sum_{k=#1}^\infty}
\newcommand{\summ}[1][1]{\sum_{m=#1}^\infty}
\newcommand{\sumn}[1][1]{\sum_{n=#1}^\infty}
\newcommand{\emp}{\varnothing}
\newcommand{\exc}{\backslash}
\newcommand{\sub}{\subseteq}
\newcommand{\sups}{\supseteq}
\newcommand{\capp}{\bigcap}
\newcommand{\cupp}{\bigcup}
\newcommand{\kupp}{\bigsqcup}
\newcommand{\cappkn}{\bigcap_{k=1}^n}
\newcommand{\cuppkn}{\bigcup_{k=1}^n}
\newcommand{\kuppkn}{\bigsqcup_{k=1}^n}
\newcommand{\cappk}[1][1]{\bigcap_{k=#1}^\infty}
\newcommand{\cuppk}[1][1]{\bigcup_{k=#1}^\infty}
\newcommand{\cappm}[1][1]{\bigcap_{m=#1}^\infty}
\newcommand{\cuppm}[1][1]{\bigcup_{m=#1}^\infty}
\newcommand{\cappn}[1][1]{\bigcap_{n=#1}^\infty}
\newcommand{\cuppn}[1][1]{\bigcup_{n=#1}^\infty}
\newcommand{\kuppk}[1][1]{\bigsqcup_{k=#1}^\infty}
\newcommand{\kuppm}[1][1]{\bigsqcup_{m=#1}^\infty}
\newcommand{\kuppn}[1][1]{\bigsqcup_{n=#1}^\infty}
\newcommand{\cappa}{\bigcap_{\alpha\in I}}
\newcommand{\cuppa}{\bigcup_{\alpha\in I}}
\newcommand{\kuppa}{\bigsqcup_{\alpha\in I}}
\newcommand{\Rx}{\overline{\mathbb{R}}}
\newcommand{\dx}{\,dx}
\newcommand{\dy}{\,dy}
\newcommand{\dt}{\,dt}
\newcommand{\dax}{\,d\alpha(x)}
\newcommand{\dbx}{\,d\beta(x)}
\DeclareMathOperator{\glb}{\text{glb}}
\DeclareMathOperator{\lub}{\text{lub}}
\newcommand{\xh}{\widehat{x}}
\newcommand{\yh}{\widehat{y}}
\newcommand{\zh}{\widehat{z}}
\newcommand{\<}{\langle}
\renewcommand{\>}{\rangle}
\renewcommand{\iff}{\Leftrightarrow}
\DeclareMathOperator{\im}{\text{im}}
\let\spn\relax\let\Re\relax\let\Im\relax
\DeclareMathOperator{\spn}{\text{span}}
\DeclareMathOperator{\sym}{\text{Sym}}
\DeclareMathOperator{\myskew}{\text{Skew}}
\DeclareMathOperator{\Re}{\text{Re}}
\DeclareMathOperator{\Im}{\text{Im}}
\DeclareMathOperator{\diag}{\text{diag}}
\endinput

% Insert your name here!
\scribe{Student Name: Noah Reef}
\newcommand{\rb}{\bold{r}}

\begin{document}
\maketitle

\section*{Problem 1}
\subsection*{Part a}
\begin{equation*}
    L(
\end{equation*}
\section*{Problem 2}
\subsection*{Part a}
\begin{equation*}
  L(x, \dot{x}, t) = T - V = \frac{1}{2}m\dot{x}^2 - \frac{1}{2}kx^2
\end{equation*}
\subsection*{Part b}
Using the Euler-lagrange equation, we have:
\begin{equation*}
  \frac{\partial L}{\partial x} - \frac{d}{dt}\left(\frac{\partial L}{\partial \dot{x}}\right) = -kx - m \ddot{x} = 0
\end{equation*}
solving the differential equation above gives the general solution,
\begin{equation*}
  x(t) = c_2 \sin\left(\omega t\right) + c_1 \cos\left(\omega t\right)
\end{equation*}
where $\omega = \sqrt{\frac{k}{m}}$. Then we see with boundary equations that
\begin{align*}
  x(0) &= c_1 = x_1 \\
  x(\tau) &= c_2 \sin\left(\omega \tau\right) + x_1 \cos\left(\omega \tau\right) = x_2
\end{align*}
solving for $c_2$ gives
\begin{equation*}
  c_2 = \frac{x_2 - x_1 \cos\left(\omega \tau\right)}{\sin\left(\omega \tau\right)}
\end{equation*}
since $\tau \neq \frac{n\pi}{\omega}$, we have that $\sin(\omega \tau) \neq 0$. 

\subsection*{Part c}
To solve for the action $S$, we have that
\begin{align*}
  S &= \int_0^\tau L(x, \dot{x}, t) dt \\
    &= \int_0^\tau \left(\frac{1}{2}m\dot{x}^2 - \frac{1}{2}kx^2\right) dt \\
    &= \frac{1}{2}m\int_0^\tau \dot{x}^2 - \omega^2 x^2 \, dt \\ 
\end{align*}
Now notice that
\begin{align*}
  \int_0^\tau \dot{x}^2 \, dt &= x\dot{x}\Big|_0^\tau - \int_0^\tau x\ddot{x} \, dt 
\end{align*}
and since $\ddot{x} = -\omega^2 x$, we have that
\begin{equation*}
  \int_0^\tau \dot{x}^2 \, dt = x\dot{x}\Big|_0^\tau + \omega^2 \int_0^\tau x^2 \, dt
\end{equation*}
Thus, we have that
\begin{equation*}
  S = \frac{1}{2}m\left(x\dot{x}\Big|_0^\tau + \omega^2 \int_0^\tau x^2 \, dt - \omega^2 \int_0^\tau x^2 \, dt \right)
\end{equation*}
thus
\begin{equation*}
  S = \frac{1}{2}m\left(x\dot{x}\Big|_0^\tau\right)
\end{equation*}
to solve the inside term we have that
\begin{align*}
  x\dot{x} &= \left(c_2 \sin(\omega t) + x_1 \cos(\omega t)\right) (\omega c_2 \cos(\omega t) - x_1 \omega \sin(\omega t)) 
\end{align*}
now evaluating at $\tau$ and $0$ gives
\begin{align*}
  x\dot{x}\Big|_0^\tau &= \left(c_2 \sin(\omega \tau) + x_1 \cos(\omega \tau)\right) (\omega c_2 \cos(\omega \tau) - x_1 \omega \sin(\omega \tau))-  x_1\omega c_2 \\
                       &= \left(\frac{x_2 - x_1\cos(\omega \tau)}{\sin(\omega \tau)} \sin(\omega \tau) + x_1\cos(\omega \tau)\right) \left(\frac{\omega x_2\cos(\omega \tau)-\omega x_1 \cos^2(\omega \tau)}{\sin(\omega \tau)} - x_1 \omega \sin(\omega \tau)\right) - \dots \\
                       &= \frac{\omega x_2^2 \cos(\omega \tau) - \omega x_2x_1}{\sin(\omega \tau)} - \frac{\omega x_1x_2 - \omega x_1^2 \cos(\omega \tau)}{\sin(\omega \tau)} \\
                       &= \frac{\omega}{\sin(\omega \tau)}(-2x_1x_2 + (x_1^2 + x_2^2)\cos(\omega \tau))
\end{align*}
thus 
\begin{equation*}
  S = \frac{m\omega}{2\sin(\omega \tau)}(-2x_1x_2 + (x_1^2 + x_2^2)\cos(\omega \tau))
\end{equation*}

\subsection*{Part d}
To write the action $S(x,\tau)$, we set $x_2 = x$ and $x_1 = 0$ to get 
\begin{equation*}
  S(x,\tau) = \frac{m\omega}{2\sin(\omega \tau)}(-2x + x^2\cos(\omega \tau))
\end{equation*}
to solve for the action of a silicon atom, with mass $m_{\text{Si}} = 4.66 \times 10^{-26} \si{kg}$, vibrating at frequecy $\omega = 15\si{THz}$ and amplitude $x = a_0$ and a time $\tau = \pi/(4\omega)$
\end{document}
