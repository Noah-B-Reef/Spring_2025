\documentclass[12pt]{report}
\usepackage{scribe,graphicx,graphics}
\usepackage{float}
\usepackage{siunitx}
\course{CSE 389D} 	
\coursetitle{Mathematical Modeling}	
\semester{Spring 2025}
\lecturer{} % Due Date: {\bf Mon, Oct 3 2016}}
\lecturetitle{Problem Set}
\lecturenumber{2}   
\lecturedate{}    
\input{commands.tex}

\theoremheaderfont{\itshape}
\theorembodyfont{\upshape}
\newtheorem{case}{Case}

% Insert your name here!
\scribe{Student Name: Noah Reef}
\newcommand{\rb}{\bold{r}}

\begin{document}
\maketitle
\section*{Problem 2.1}
\subsection*{Part a}
\begin{figure}[H]
    \centering
    \includegraphics[scale=0.7]{Figure_1.png}
    \caption{Sketch of $V(x)$ for $a=1$,$b=3$, and $V_0 = 10$}
    \label{fig:enter-label}
\end{figure}
The time-independent Schrodinger's equation is given by
\begin{equation*}
    -\frac{\hbar^2}{2m} \frac{d^2\phi(x)}{dx^2} + V(x) \phi(x) = E\phi(x)
\end{equation*}

\subsection*{Part b}
If we consider the bounded state where $-V_0 < E < 0$, and $0 < x < a$, then we get that $V(x) = 0$ and
\begin{equation*}
    -\frac{\hbar^2}{2m} \phi''(x) = E \phi(x) \implies \phi''(x) = \frac{2m |E|}{\hbar^2} \phi(x)
\end{equation*}
solving the differential equation above gives the general solution,
\begin{equation*}
    \phi(x) = Ae^{\kappa x} + Be^{-\kappa x}
\end{equation*}
where
\begin{equation*}
    \kappa = \sqrt{\frac{2m |E|}{\hbar^2}}
\end{equation*}
however since it is symmetric we get that
\begin{equation*}
     \phi(x) = Ae^{\kappa x} + Ae^{-\kappa x} = 2A\cosh(\kappa x)
\end{equation*}
now if we consider the case where $a \leq x \leq b$, we have that $V(x) - V_0$ and get
\begin{equation*}
    -\frac{\hbar^2}{2m}\phi''(x) - V_0 \phi(x) = E\phi(x) \implies \phi''(x) = -\frac{2m(E + V_0)}{\hbar^2} \phi(x)
\end{equation*}
solving the above differential equation yields the following general solution
\begin{equation*}
    \phi(x) = C \cos(\xi (b-x)) + D \sin(\xi(b-x))
\end{equation*}
where 
\begin{equation*}
    \xi = \sqrt{\frac{2m(E + V_0)}{\hbar^2}}
\end{equation*}
then since $V(b) = \infty$ we require that $\phi(b) = 0$ and hence 
\begin{equation*}
    \phi(x) = D \sin(\xi (b-x))
\end{equation*}
now $x = a$ we have that
\begin{equation*}
    2A \cosh(\kappa a) = D \sin(\xi(b-a))
\end{equation*}
taking the derivatives 
\begin{equation*}
    2A\kappa \sinh(\kappa a) =  -D\xi\cos(\xi(b-a))
\end{equation*}
then dividing both equations yields
\begin{equation*}
    \frac{\kappa \sinh(\kappa a)}{\cosh(\kappa a)} = -\frac{\xi \cos(\xi(b-a))}{\sin(\xi(b-a))} \implies \kappa \tanh(\kappa a) = -\xi \cot(\xi (b-a)) = -\xi \frac{1 + \tan(\xi a)\tan(\xi b)}{\tan(\xi b) - \tan(\xi a)}
\end{equation*}
thus
\begin{equation*}
\kappa \tanh(\kappa a) + \xi \frac{1 + \tan(\xi a)\tan(\xi b)}{\tan(\xi b) - \tan(\xi a)} = 0
\end{equation*}
letting $v = \xi b$ and $a = \gamma b$ then we get that
\begin{equation*}
    \kappa \tanh(\kappa a) + (v/b) \frac{1 + \tan(\gamma v)\tan(v)}{\tan(v) - \tan(\gamma v)} = 0
\end{equation*}
note that if we define
\begin{equation*}
    S = \frac{b\sqrt{2mV_0}}{\hbar}
\end{equation*}
then we have that
\begin{align*}
    \kappa = \sqrt{\frac{2m|E|}{\hbar^2}} =\sqrt{\frac{2m(V_0 - (E + V_0)}{\hbar^2}} &= \sqrt{\frac{2mV_0}{\hbar^2} -\xi^2} \\
    &= \sqrt{\frac{2mV_0}{\hbar^2} -\frac{v^2}{b^2}} \\
    &= \frac{1}{b} \sqrt{S^2 - v^2}
\end{align*}
therefore we have that
\begin{equation*}
   \sqrt{S^2 - v^2} \tanh(\gamma \sqrt{S^2 - v^2}) - v \frac{1 + \tan(\gamma v)\tan(v)}{\tan(\gamma v) - \tan(v)} = 0
\end{equation*}
and hence our eigenvalues are given by
\begin{equation*}
    E + V_0 = \frac{\hbar^2 \xi^2}{2m} = \frac{\hbar^2 v^2}{2mb^2} \implies \frac{E}{V_0} = -1 + \frac{v^2}{S^2}
\end{equation*}
\subsection*{Part c}
Solving for $v$ from the equation below we get
\begin{align*}
    v_1 &= 9.87725 \\
    v_2 &= 6.84406 \\
    v_3 &= 3.46525 
\end{align*}
plugging these values into the equation
\begin{equation*}
    \frac{E}{V_0} = -1 + \frac{v^2}{S^2} 
\end{equation*}
yields
\begin{align*}
     \frac{E}{V_0} &= -1 + \frac{(9.87725)^2}{S^2} \approx -0.0244\\
      \frac{E}{V_0} &= -1 + \frac{(6.84406)^2}{S^2} \approx -0.5316\\
       \frac{E}{V_0} &= -1 + \frac{(3.46525)^2}{S^2} \approx -0.8799
\end{align*}
as the eigenvalues of the even bounded state problem.


\subsection*{Part d}
\subsection*{Part e}
To show the odd parity of the wave function we will get a similar result as in part b. However we notice that since the wave is odd we we get that in the case $|x| < a$ the wave function is given by
\begin{equation*}
  \phi(x) = Ae^{\kappa x} - Ae^{-\kappa x} = 2A\sinh(\kappa x)
\end{equation*}
and in the case $a < x < b$ we have that
\begin{equation*}
    \phi(x) = D \sin(\xi (b-x))
\end{equation*}
as before. Then we have that
\begin{align*}
  2A\sinh(\kappa a) &= D \sin(\xi(b-a)) \\
  2A\kappa \cosh(\kappa a) &= -D\xi\cos(\xi(b-a))
\end{align*}
then dividing the two equations yields
\begin{equation*}
    \frac{\kappa \cosh(\kappa a)}{\sinh(\kappa a)} = -\frac{\xi \cos(\xi(b-a))}{\sin(\xi(b-a))} \implies \kappa \coth(\kappa a) = -\xi \cot(\xi(b-a))
\end{equation*}
then by doing similar algebra and subsitutions as in part b we get that
\begin{equation*}
  \frac{\sqrt{S^2 - v^2}}{\tanh(\gamma \sqrt{S^2 - v^2})} - v \frac{\tan(\gamma v)\tan(v) + 1}{\tan(\gamma v) - \tan(v)} = 0
\end{equation*}
and hence the eigenvalues are given by 
\begin{equation*}
  \frac{E}{V_0} = -1 + \frac{v^2}{S^2}
\end{equation*}
as desired.

\subsection*{Part f}
Solving for $v$ from the equation below, using $S = 10$ and $\gamma = 0.2$ we get that
\begin{align*}
  v_1 &=   6.96192 \\
  v_2 &= 3.49913 \\
\end{align*}
plugging these values into the equation gives the following eigenvalues
\begin{align*}
  \frac{E}{V_0} &= -1 + \frac{(6.96192)^2}{S^2} \approx -0.5153\\
  \frac{E}{V_0} &= -1 + \frac{(3.49913)^2}{S^2} \approx -0.8775
\end{align*}

\subsection*{Part g}
The parity of the ground-state wavefunction is even.

\subsection*{Part h}

\section*{Problem 2.2}

\end{document}
