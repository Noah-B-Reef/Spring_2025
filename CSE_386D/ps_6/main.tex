\documentclass[12pt]{report}
\usepackage{scribe,graphicx,graphics}
\usepackage{bbm}
\usepackage{yhmath}
\usepackage{stackengine}
\newcommand\dhookrightarrow{\mathrel{%
  \ensurestackMath{\stackanchor[.1ex]{\hookrightarrow}{\hookrightarrow}}
}}
\newcommand{\norm}[1]{\left|\left|#1\right|\right|}
\newcommand{\inner}[2]{\left\langle#1,#2\right\rangle}

\course{CSE 386D} 	
\coursetitle{Methods of Applied Mathematics II}	
\semester{Spring 2025}
\lecturer{} % Due Date: {\bf Mon, Oct 3 2016}}
\lecturetitle{Problem Set}
\lecturenumber{6}   
\lecturedate{}    
\usepackage{enumerate}
\newcommand{\remind}[1]{\textcolor{red}{\textbf{#1}}} %To remind me of unfinished work to fix later
\newcommand{\hide}[1]{} %To hide large blocks of code without using % symbols

\newcommand{\ep}{\varepsilon}
\newcommand{\vp}{\varphi}
\newcommand{\lam}{\lambda}
\newcommand{\Lam}{\Lambda}
%\newcommand{\abs}[1]{\ensuremath{\left\lvert#1\right\rvert}} % This clashes with the physics package
%\newcommand{\norm}[1]{\ensuremath{\left\lVert#1\right\rVert}} % This clashes with the physics package
\newcommand{\floor}[1]{\ensuremath{\left\lfloor#1\right\rfloor}}
\newcommand{\ceil}[1]{\ensuremath{\left\lceil#1\right\rceil}}
\newcommand{\A}{\mathbb{A}}
\newcommand{\B}{\mathbb{B}}
\newcommand{\C}{\mathbb{C}}
\newcommand{\D}{\mathbb{D}}
\newcommand{\E}{\mathbb{E}}
\newcommand{\F}{\mathbb{F}}
\newcommand{\K}{\mathbb{K}}
\newcommand{\N}{\mathbb{N}}
\newcommand{\Q}{\mathbb{Q}}
\newcommand{\R}{\mathbb{R}}
\newcommand{\T}{\mathbb{T}}
\newcommand{\X}{\mathbb{X}}
\newcommand{\Y}{\mathbb{Y}}
\newcommand{\Z}{\mathbb{Z}}
\newcommand{\As}{\mathcal{A}}
\newcommand{\Bs}{\mathcal{B}}
\newcommand{\Cs}{\mathcal{C}}
\newcommand{\Ds}{\mathcal{D}}
\newcommand{\Es}{\mathcal{E}}
\newcommand{\Fs}{\mathcal{F}}
\newcommand{\Gs}{\mathcal{G}}
\newcommand{\Hs}{\mathcal{H}}
\newcommand{\Is}{\mathcal{I}}
\newcommand{\Js}{\mathcal{J}}
\newcommand{\Ks}{\mathcal{K}}
\newcommand{\Ls}{\mathcal{L}}
\newcommand{\Ms}{\mathcal{M}}
\newcommand{\Ns}{\mathcal{N}}
\newcommand{\Os}{\mathcal{O}}
\newcommand{\Ps}{\mathcal{P}}
\newcommand{\Qs}{\mathcal{Q}}
\newcommand{\Rs}{\mathcal{R}}
\newcommand{\Ss}{\mathcal{S}}
\newcommand{\Ts}{\mathcal{T}}
\newcommand{\Us}{\mathcal{U}}
\newcommand{\Vs}{\mathcal{V}}
\newcommand{\Ws}{\mathcal{W}}
\newcommand{\Xs}{\mathcal{X}}
\newcommand{\Ys}{\mathcal{Y}}
\newcommand{\Zs}{\mathcal{Z}}
\newcommand{\ab}{\textbf{a}}
\newcommand{\bb}{\textbf{b}}
\newcommand{\cb}{\textbf{c}}
\newcommand{\db}{\textbf{d}}
\newcommand{\ub}{\textbf{u}}
\newcommand{\sbb}{\textbf{s}}
%\renewcommand{\vb}{\textbf{v}} % This clashes with the physics package (the physics package already defines the \vb command)
\newcommand{\wb}{\textbf{w}}
\newcommand{\xb}{\textbf{x}}
\newcommand{\yb}{\textbf{y}}
\newcommand{\zb}{\textbf{z}}
\newcommand{\vbb}{\textbf{v}}
\newcommand{\Ab}{\textbf{A}}
\newcommand{\Bb}{\textbf{B}}
\newcommand{\Cb}{\textbf{C}}
\newcommand{\Db}{\textbf{D}}
\newcommand{\eb}{\textbf{e}}
\newcommand{\ex}{\textbf{e}_x}
\newcommand{\ey}{\textbf{e}_y}
\newcommand{\ez}{\textbf{e}_z}
\newcommand{\zerob}{\mathbf{0}}
\newcommand{\abar}{\overline{a}}
\newcommand{\bbar}{\overline{b}}
\newcommand{\cbar}{\overline{c}}
\newcommand{\dbar}{\overline{d}}
\newcommand{\ubar}{\overline{u}}
\newcommand{\vbar}{\overline{v}}
\newcommand{\wbar}{\overline{w}}
\newcommand{\xbar}{\overline{x}}
\newcommand{\ybar}{\overline{y}}
\newcommand{\zbar}{\overline{z}}
\newcommand{\Abar}{\overline{A}}
\newcommand{\Bbar}{\overline{B}}
\newcommand{\Cbar}{\overline{C}}
\newcommand{\Dbar}{\overline{D}}
\newcommand{\Ubar}{\overline{U}}
\newcommand{\Vbar}{\overline{V}}
\newcommand{\Wbar}{\overline{W}}
\newcommand{\Xbar}{\overline{X}}
\newcommand{\Ybar}{\overline{Y}}
\newcommand{\Zbar}{\overline{Z}}
\newcommand{\Aint}{A^\circ}
\newcommand{\Bint}{B^\circ}
\newcommand{\limk}{\lim_{k\to\infty}}
\newcommand{\limm}{\lim_{m\to\infty}}
\newcommand{\limn}{\lim_{n\to\infty}}
\newcommand{\limx}[1][a]{\lim_{x\to#1}}
\newcommand{\liminfm}{\liminf_{m\to\infty}}
\newcommand{\limsupm}{\limsup_{m\to\infty}}
\newcommand{\liminfn}{\liminf_{n\to\infty}}
\newcommand{\limsupn}{\limsup_{n\to\infty}}
\newcommand{\sumkn}{\sum_{k=1}^n}
\newcommand{\sumk}[1][1]{\sum_{k=#1}^\infty}
\newcommand{\summ}[1][1]{\sum_{m=#1}^\infty}
\newcommand{\sumn}[1][1]{\sum_{n=#1}^\infty}
\newcommand{\emp}{\varnothing}
\newcommand{\exc}{\backslash}
\newcommand{\sub}{\subseteq}
\newcommand{\sups}{\supseteq}
\newcommand{\capp}{\bigcap}
\newcommand{\cupp}{\bigcup}
\newcommand{\kupp}{\bigsqcup}
\newcommand{\cappkn}{\bigcap_{k=1}^n}
\newcommand{\cuppkn}{\bigcup_{k=1}^n}
\newcommand{\kuppkn}{\bigsqcup_{k=1}^n}
\newcommand{\cappk}[1][1]{\bigcap_{k=#1}^\infty}
\newcommand{\cuppk}[1][1]{\bigcup_{k=#1}^\infty}
\newcommand{\cappm}[1][1]{\bigcap_{m=#1}^\infty}
\newcommand{\cuppm}[1][1]{\bigcup_{m=#1}^\infty}
\newcommand{\cappn}[1][1]{\bigcap_{n=#1}^\infty}
\newcommand{\cuppn}[1][1]{\bigcup_{n=#1}^\infty}
\newcommand{\kuppk}[1][1]{\bigsqcup_{k=#1}^\infty}
\newcommand{\kuppm}[1][1]{\bigsqcup_{m=#1}^\infty}
\newcommand{\kuppn}[1][1]{\bigsqcup_{n=#1}^\infty}
\newcommand{\cappa}{\bigcap_{\alpha\in I}}
\newcommand{\cuppa}{\bigcup_{\alpha\in I}}
\newcommand{\kuppa}{\bigsqcup_{\alpha\in I}}
\newcommand{\Rx}{\overline{\mathbb{R}}}
\newcommand{\dx}{\,dx}
\newcommand{\dy}{\,dy}
\newcommand{\dt}{\,dt}
\newcommand{\dax}{\,d\alpha(x)}
\newcommand{\dbx}{\,d\beta(x)}
\DeclareMathOperator{\glb}{\text{glb}}
\DeclareMathOperator{\lub}{\text{lub}}
\newcommand{\xh}{\widehat{x}}
\newcommand{\yh}{\widehat{y}}
\newcommand{\zh}{\widehat{z}}
\newcommand{\<}{\langle}
\renewcommand{\>}{\rangle}
\renewcommand{\iff}{\Leftrightarrow}
\DeclareMathOperator{\im}{\text{im}}
\let\spn\relax\let\Re\relax\let\Im\relax
\DeclareMathOperator{\spn}{\text{span}}
\DeclareMathOperator{\sym}{\text{Sym}}
\DeclareMathOperator{\myskew}{\text{Skew}}
\DeclareMathOperator{\Re}{\text{Re}}
\DeclareMathOperator{\Im}{\text{Im}}
\DeclareMathOperator{\diag}{\text{diag}}
\endinput
\newcommand{\supp}[]{\text{supp}}

% Insert yousupp}
\scribe{Student Name: Noah Reef}

\begin{document}
\maketitle

\section*{Problem 7.9}
Let $\Omega \subseteq \R^d$ be bounded and suppose that $f_j \in H^2(\Omega)$ is such that $f_j \rightharpoonup f$ in $H^1(\Omega)$ and $D^{\alpha} f_j \rightharpoonup g_{\alpha}$ in $L^2(\Omega)$ for all $|\alpha| = 2$. Without loss of generality, we will assume that $\Omega$ is Lipschitz, since we can always find a bounded extension operator from $\Omega$ to a Lipschitz domain and the results will still hold. 

Since $f_j \in H^2(\Omega)$, we have that $\norm{f_j}_{H^2(\Omega)} < \infty$ and 
thus we have that there exists a subsequence $\{f_{j_k}\}_{k=1}^\infty$ such that $f_{j_k} \to f_{H^2}$ in $H^2(\Omega)$. Additionally, we have that since $f_j \rightharpoonup f$ in $H^1(\Omega)$, then we have that $\norm{f_j}_{H^1(\Omega)}$ is bounded and thus there exists a subsequence $\{f_{j_k}\}_{k=1}^\infty$ such that $f_{j_k} \rightharpoonup f$ in $H^1(\Omega)$. Since $H^2 \hookrightarrow H^1$ and we have that the weak limit is unique, we get that $f_{j_k} \rightharpoonup f$ in $H^2(\Omega)$. Thus we have that 
\begin{align*}
  \inner{f_{j_k}}{h}_{H^1} &= \sum_{|\alpha| \leq 2} \inner{D^{\alpha} f_{j_k}}{D^{\alpha} h}_{L^2} \to \sum_{|\alpha| \leq 2} \inner{D^{\alpha}f}{D^{\alpha} h}_{L^2} = \inner{f}{h}_{H^2}
\end{align*}
which implies that
\begin{equation*}
  \inner{D^{\alpha}f_{j_k}}{D^{\alpha}h}_{L^2} \to \inner{D^{\alpha}f}{h}_{L^2}
\end{equation*}
but since $D^{\alpha}f_{j_k} \rightharpoonup g_{\alpha}$ in $L^2(\Omega)$, we have that $g_{\alpha} = D^{\alpha}f$ and hence $f \in H^2(\Omega)$. Additionally since $H^2(\Omega) \hookrightarrow H^1(\Omega)$, we have that $f_{j} \rightharpoonup f$ in $H^2(\Omega)$ and thus there exists a subsequence $f_{j_k} \to f$ in $H^1(\Omega)$.


\section*{Problem 7.10}
Suppose that $\Omega \subset \R^d$ is bounded with a Lipschitz boundary and $f_j \rightharpoonup f$ and $g_j \rightharpoonup g$ in $H^1(\Omega)$. Note that for $\varphi \in \mathcal{D}(\Omega)$, we have that
\begin{align*}
  \inner{\nabla(f_jg_j)}{\varphi} &= \int_{\Omega} g_j \nabla f_j \cdot \varphi \,dx  + \int_{\Omega} f_j \nabla g_j \cdot  \varphi \, dx \\
                                  &= \int_{\Omega} (g_j - g)\nabla f_j \cdot \varphi \, dx + \int_{\Omega} g \nabla f_j \cdot \varphi \, dx + \int_{\Omega} (f_j - f)\nabla g_j \cdot \varphi \, dx + \int_{\Omega} f \nabla g_j \cdot \varphi \, dx \\ 
                                  &= \inner{(g_j - g)\nabla f_j  }{\varphi}_{L^2(\Omega)} + \inner{g\nabla f_j}{\varphi}_{L^2(\Omega)} + \inner{(f_j - f)\nabla g_j}{\varphi}_{L^2(\Omega)} + \inner{f \nabla g_j}{\varphi}_{L^2(\Omega)} \\
\end{align*}
then by Corollary 7.23, we have that there exists $f_{j_k} \to f$ and $g_{j_k} \to g$ in $L^2(\Omega)$ and thus by using the subsequences above we get that
\begin{align*}  
  \inner{(g_{j_k} - g)\nabla f_{j_k}  }{\varphi}_{L^2(\Omega)} + \inner{g\nabla f_{j_k}}{\varphi}_{L^2(\Omega)} + \inner{(f_{j_k} - f)\nabla g_{j_k}}{\varphi}_{L^2(\Omega)} + \inner{f \nabla g_{j_k}}{\varphi}_{L^2(\Omega)} 
\end{align*}
which becomes $\inner{g \nabla f_{j_k} + f \nabla g_{j_k}}{\varphi}_{L^2(\Omega)} \to \inner{\nabla(fg)}{\varphi}$ and hence we have that there is a subsequence such that $\nabla(f_{j_k}g_{j_k}) \to \nabla(fg)$ in $L^2(\Omega)$. To have the sequence weakly converge in $L^p(\Omega)$, we require that the sequence is uniformly bounded in $L^p(\Omega)$ and so we first note that by Holder's Inequality, we have that
\begin{align*}  \norm{f_j \nabla g_j + g_j \nabla f_j}_{L^p} &\leq \norm{f_j \nabla g_j}_{L^p} + \norm{g_j \nabla f_j}_{L^p} \\
  &\leq \norm{f_j}_{L^q} \norm{\nabla g_j}_{L^2} + \norm{g_j}_{L^q} \norm{\nabla f_j}_{L^2} 
\end{align*}
where $1/p = 1/q + 1/2$. Note that since $f_j,g_j \in H^1(\Omega)$ we get that the sequence $\norm{\nabla g_j}_{L^2}$ and $\norm{\nabla f_j}_{L^2}$ are bounded. 

In the case of $d \geq 3$, we have that for $p^* = 2d/(d-2)$ that $H^1(\Omega) \hookrightarrow L^{p^*}(\Omega)$. Then by taking $q = p^*$ we have that
\begin{equation*}
  \frac{1}{p} = \frac{d-2}{2d} + \frac{1}{2} = \frac{d-2 + 2d}{2d} = \frac{d-1}{d}
\end{equation*}
and thus for $d \geq 3$ we have weak convergence in $L^{d/(d-1)}(\Omega)$. Next for $d = 2$, we have that $H^1(\Omega) \hookrightarrow L^{\infty}(\Omega)$ and thus we have that $q = \infty$ and thus we have that
\begin{equation*}
  \frac{1}{p} = 0 + \frac{1}{2} = \frac{1}{2}
\end{equation*}
and hence we have weak convergence in $L^2(\Omega)$.
\section*{Problem 7.11}
\subsection*{Part a}
Since $\Omega \subset \R^d$ is bounded with Lipschitz boundary, and $\{u_j\} \subseteq H^{2 + \epsilon}(\Omega) = W^{2+\epsilon, 2}(\Omega)$, we have that there exists a subsequence $\{u_{j_k}\} \subseteq W^{2,q}(\Omega)$ for $q < 2 < 2d/(d - 2\epsilon)$ that converges. Thus by the Rellich-Kondrachov Theorem we have that $W^{2,q}(\Omega) \hookrightarrow W^{2,2}(\Omega) = H^2(\Omega)$ and hence we have that there exists a subsequence $\{u_{j_k}\} \subseteq H^2(\Omega)$ that converges. 

\subsection*{Part b}
To find such $q$ and $s \geq 0$ such that we have $u_{j_k} \to u$ in $W^{s,q}(\Omega)$ we need that $W^{2 + \epsilon,2}(\Omega) \dhookrightarrow W^{s,q}(\Omega)$. This means that $s + m = 2 + \epsilon$ with $q < 2d/(d-2m)$, thus we have 
\begin{align*}
  q < \frac{2d}{d - 2(2 + \epsilon - s)} \implies s < 2 + \epsilon + \frac{d}{q} - \frac{d}{2}
\end{align*}
and we have that
\begin{equation*}

\section*{Part c}
Suppose we have a subsequence $|u_{j_k}|^r \nabla u_{j_k} \to |u|^r\nabla u$ in $L^2(\Omega)$ for some $r \geq 1$. Then we see that
\begin{align*}
  \norm{|u_{j_k}|^r \nabla u_{j_k} -|u|^r \nabla u}_{L^2} &= \norm{\left(|u_{j_k}|^r - |u|^r\right) \nabla u_{j_k} -|u|^r (\nabla u - \nabla u_{j_k})}_{L^2} \\
                                                          &\leq \norm{\left(|u_{j_k}|^r - |u|^r\right) \nabla u_{j_k}}_{L^2}  + \norm{|u|^r (\nabla u - \nabla u_{j_k})}_{L^2}
\end{align*}
then by Holder's Inequality we have that
\begin{align*}
  \norm{\left(|u_{j_k}|^r - |u|^r\right) \nabla u_{j_k}}_{L^2}  + \norm{|u|^r (\nabla u - \nabla u_{j_k})}_{L^2} &\leq \norm{|u_{j_k}|^r - |u|^r}_{L^{\infty}} \norm{\nabla u_{j_k}}_{L^2} + \norm{|u|^r}_{L^\infty} \norm{\nabla u - \nabla u_{j_k}}_{L^2} 
\end{align*}
since $[u_{j_k}]$ is bounded in $W^{2,2}(\Omega)$, we have that $\norm{\nabla u_{j_k}}_{L^2}$ is bounded, similarly 
\begin{equation*}
  \norm{\nabla u - \nabla u_{j_k}}_{L^2} \to 0
\end{equation*}
and thus we need to check if $\norm{|u_{j_k}|^r - |u|^r}_{L^{\infty}} \to 0$, which requires that $H^{2}(\Omega) \hookrightarrow L^{\infty}(\Omega)$. Note we have the following inequality,
\begin{equation*}
  s < 2 + \frac{d}{q} - \frac{d}{2}
\end{equation*}
and thus for $d \leq 4$ choosing $s = 0$ and $q = \infty$ satisfies that above inequality and hence $\norm{|u_{j_k}|^r - |u|^r}_{L^{\infty}} \to 0$ and thus we have that $|u_{j_k}|^r \nabla u_{j_k} \to |u|^r \nabla u$ in $L^2(\Omega)$ for any $r \geq 1$. For $d > 4$ we could have instead applied Holder's as
\begin{align*}
  \norm{\left(|u_{j_k}|^r - |u|^r\right) \nabla u_{j_k}}_{L^2}  + \norm{|u|^r (\nabla u - \nabla u_{j_k})}_{L^2} &\leq \norm{|u_{j_k}|^r - |u|^r}_{L^{s}} \norm{\nabla u_{j_k}}_{L^t} + \norm{|u|^r}_{L^s} \norm{\nabla u - \nabla u_{j_k}}_{L^t} 
\end{align*}
and here we make the choice that $s = \frac{2d}{d-4}$ and $t = d/2$ to see that $H^2(\Omega) \hookrightarrow L^t(\Omega)$ and thus we have that $\nabla u_{j_k}$ converges in $L^t(\Omega)$. Then lastly

\begin{equation*}
  \norm{|u_{j_k}|^r - |u|^r}_{L^{s}} \to 0
\end{equation*}
which we can achieve for $r \leq \frac{2}{d-4}$

  \end{document}
