\documentclass[12pt]{report}
\usepackage{scribe,graphicx,graphics}
\usepackage{bbm}
\usepackage{yhmath}
\newcommand{\norm}[1]{\left|\left|#1\right|\right|}
\newcommand{\inner}[2]{\left\langle#1,#2\right\rangle}

\course{CSE 386D} 	
\coursetitle{Methods of Applied Mathematics II}	
\semester{Spring 2025}
\lecturer{} % Due Date: {\bf Mon, Oct 3 2016}}
\lecturetitle{Problem Set}
\lecturenumber{3}   
\lecturedate{}    
\input{commands.tex}

% Insert your name here!
\scribe{Student Name: Noah Reef}

\begin{document}
\maketitle

\section*{Problem 6.16}
Consider 
\begin{equation*}
    u = \sum_{k=1}^\infty a_k \delta_k
\end{equation*}
then for any $\phi \in \mathcal{S}$, we have that
\begin{align*}
    \inner{u}{\phi} = \int \left(\sum_{k=1}^\infty a_k \delta_k(x) \right) \phi(x) \, dx &= \sum_{k=1}^\infty a_k \int \delta_k(x) \phi(x) \, dx \\
    &= \sum_{k=1}^\infty a_k \inner{\delta_k}{\phi} \\
    &= \sum_{k=1}^\infty a_k \phi(k)
\end{align*}
Note that since $\phi \in \mathcal{S}$, we have that 
\begin{equation*}
    |u(\phi)| \leq \sum_{k=1}^\infty |a_k \phi(k)| \leq \norm{\phi}_\infty \sum_{k=1}^\infty |a_k|
\end{equation*}
and hence require that series of $a_k$ to converge absolutely for $u$ to be in $\mathcal{S}'$.

\section*{Problem 6.17}
Let $f \in L^2(\R)$ and define the Hilbert transform of $f$ by
\begin{equation*}
    Hf = \text{PV}\left(\frac{1}{\pi x}\right)  f(x)
\end{equation*}
\subsection*{Part a}
Recall that 
\begin{equation*}
    D \hat{f}(\xi) = -i \mathcal{F}(xf(x))
\end{equation*}
and hence we have that
\begin{equation*}
    \mathcal{F}(x \text{PV}(1/x)) = i D \mathcal{F}(\text{PV}(1/x))
\end{equation*}
note that $x\text{PV}(1/x) = 1$ and $\mathcal{F}(1) = \sqrt{2\pi} \delta_0$ and therefore
\begin{equation*}
     D \mathcal{F}(\text{PV}(1/x)) = -i\sqrt{2\pi} \delta_0
\end{equation*}
Thus we have that $\mathcal{F}(\text{PV}(1/x))$ is a constant away from $\xi = 0$, since $\delta_0(\xi) = 0$, and since $\text{PV}(1/x)$ is an odd function, we have that 
\begin{equation*}
    \mathcal{F}(\text{PV}(1/x)) = C\text{sgn}(\xi)
\end{equation*}
and hence
\begin{equation*}
    D( C\text{sgn}(\xi)) = 2C\delta_0(\xi) = -i\sqrt{2\pi} \delta_0
\end{equation*}
thus $C = -i\sqrt{\pi/2}$ and hence $\mathcal{F}(\text{PV}(1/x)) = -i\sqrt{\pi/2} \text{ sgn}(\xi)$
\subsection*{Part b}
Here we see that
\begin{align*}
    \mathcal{F}(Hf) = \mathcal{F}(\text{PV}(1/\pi x) * f) &= \sqrt{2\pi} \frac{1}{\pi}\mathcal{F}(\text{PV}(1/x))\mathcal{F}(f) \\
    &= \frac{\sqrt{2\pi}}{\pi}\left(-i \sqrt{\frac{\pi}{2}} \text{sgn}(x)\right) \mathcal{F}(f) \\
    &= -i \text{sgn}(x) \mathcal{F}(f)
\end{align*}
and then we have that
\begin{equation*}
    \norm{\mathcal{F}(Hf)}_2 = \norm{Hf}_2 = \norm{\mathcal{F}(f)}_2 = \norm{f}_2
\end{equation*}
and we have that
\begin{align*}
    \mathcal{F}(H(Hf)) &= \frac{1}{\pi^2} \mathcal{F}(\text{PV}(1/x) * \text{PV}(1/x) * f) \\
    &= \frac{2\pi}{\pi^2}\left[-i \sqrt{\frac{\pi}{2}} \text{sgn}(\xi)\right]^2 \mathcal{F}(f) \\
    &= -\mathcal{F}(f)
\end{align*}
then by taking the inverse fourier transform to both sides yields
\begin{equation*}
    H(Hf) = -f
\end{equation*}

\section*{Problem 6.23}
Consider the Telegrapher's equation
\begin{equation*}
    u_{tt} + u_t + u = c^2u_{xx} \quad \text{ for $x \in \R$ and $t > 0$}
\end{equation*}
where also
\begin{equation*}
    u(x,0) = f(x) \quad \text{and} \quad u_t(x,0) = g(x)
\end{equation*}
are given in $L^2(\R)$
\subsection*{Part a}
We can solve the above by applying the Fourier Transform (w.r.t x) to both sides yielding
\begin{equation*}
    \frac{\partial^2}{\partial t^2} \hat{u}(\xi,t) + \frac{\partial }{\partial t} \hat{u}(\xi,t) + \hat{u}(\xi,t) = -c^2\xi^2 \hat{u}(\xi,t)
\end{equation*}
which is a second-order ordinary differential equation, in which solving yields
\begin{equation*}
    \hat{u}(\xi,t) = e^{-t/2}\left(A(\xi) e^{i\lambda t} + B(\xi) e^{-i\lambda t}\right)
\end{equation*}
where 
\begin{equation*}
    \lambda = \frac{\sqrt{4c^2\xi^2 + 3}}{2}
\end{equation*}
we then solve for the particular solution using the initial conditions above,
\begin{align*}
    A(\xi) + B(\xi) &= \hat{f}(\xi) \\
    -\frac{1}{2}(A(\xi) + B(\xi)) + i\lambda (A(\xi) - B(\xi)) &= \hat{g}(\xi)
\end{align*} 
and solving we get
\begin{align*}
    A(\xi) = \frac{1}{2}\left(\hat{f} + \frac{\hat{g} + \frac{1}{2}\hat{f}}{i\lambda}\right) \quad \text{and} \quad B(\xi) = \frac{1}{2}\left(\hat{f} - \frac{\hat{g} + \frac{1}{2}\hat{f}}{i\lambda}\right)
\end{align*}
and by taking the inverse fourier transform we get that,
\begin{align*}
    u(x,t) &= \frac{e^{-t/2}}{\sqrt{2\pi}} \int_{-\infty}^\infty \left[A(\xi)e^{i\lambda t} + B(\xi)e^{-i\lambda t}\right] e^{i\xi x} d\xi \\
    &= \frac{e^{-t/2}}{\sqrt{2\pi}} \int_{-\infty}^\infty \left[\frac{1}{2}\left(\hat{f} + \frac{\hat{g} + \frac{1}{2}\hat{f}}{i\lambda}\right)e^{i\lambda t} + \frac{1}{2}\left(\hat{f} - \frac{\hat{g} + \frac{1}{2}\hat{f}}{i\lambda}\right)e^{-i\lambda t}\right]e^{i\xi x} d\xi \\
    &= \frac{e^{-t/2}}{\sqrt{2\pi}} \int_{-\infty}^\infty \left[\frac{\hat{f}}{2}(e^{i\lambda t} + e^{-i\lambda t}) + \left( \frac{\hat{g} + \frac{1}{2}\hat{f}}{2i\lambda}\right)(e^{i\lambda t} - e^{-i\lambda t})\right]e^{i\xi x} d\xi \\
    &= \frac{e^{-t/2}}{\sqrt{2\pi}} \int_{-\infty}^\infty \left[\hat{f}\cos(\lambda t) + \left( \frac{\hat{g} + \frac{1}{2}\hat{f}}{\lambda}\right)\sin(\lambda t)\right]e^{i\xi x} d\xi 
\end{align*} 
\subsection*{Part b}
We'll define
\begin{align*}
    A(\xi,t) &= \hat{f}\cos(\lambda t) \\
    B(\xi,t) &=  \left( \frac{\hat{g} + \frac{1}{2}\hat{f}}{\lambda}\right)\sin(\lambda t)
\end{align*}
then we see that
\begin{equation*}
    u_t = -\frac{e^{-t/2}}{2\sqrt{2\pi}} \int_{\infty}^\infty \left[A(\xi,t) + B(\xi,t)\right]e^{ix\xi} \, d\xi + \frac{e^{-t/2}}{\sqrt{2\pi}} \frac{\partial}{\partial t}\int_{\infty}^\infty \left[A(\xi,t) + B(\xi,t)\right]e^{ix\xi} \, d\xi
\end{equation*}
and 
\begin{align*}
    u_{tt} &= \frac{e^{-t/2}}{4\sqrt{2\pi}} \int_{\infty}^\infty \left[A(\xi,t) + B(\xi,t)\right]e^{ix\xi} \, d\xi + -\frac{e^{-t/2}}{2\sqrt{2\pi}} \frac{\partial }{\partial t}\int_{\infty}^\infty \left[A(\xi,t) + B(\xi,t)\right]e^{ix\xi} \, d\xi \\
    &+ -\frac{e^{-t/2}}{2\sqrt{2\pi}} \frac{\partial}{\partial t}\int_{\infty}^\infty \left[A(\xi,t) + B(\xi,t)\right]e^{ix\xi} \, d\xi + \frac{e^{-t/2}}{\sqrt{2\pi}} \frac{\partial^2}{\partial t^2}\int_{\infty}^\infty \left[A(\xi,t) + B(\xi,t)\right]e^{ix\xi} \, d\xi
\end{align*}
then we see that
\begin{equation*}
    u_{tt} + u_t + u =  \frac{3}{4}\frac{e^{-t/2}}{\sqrt{2\pi}} \int_{\infty}^\infty \left[A(\xi,t) + B(\xi,t)\right]e^{ix\xi} \, d\xi + \frac{e^{-t/2}}{\sqrt{2\pi}} \frac{\partial^2}{\partial t^2}\int_{\infty}^\infty \left[A(\xi,t) + B(\xi,t)\right]e^{ix\xi} \, d\xi
\end{equation*}
notice that
\begin{align*}
    \frac{\partial^2}{\partial t^2} A(\xi,t) &= -\lambda^2 \hat{f} \cos(\lambda t) = -\left(c^2\xi^2 + \frac{3}{4}\right)A(\xi,t) \\
    \frac{\partial^2}{\partial t^2} B(\xi,t) &= -\lambda^2 B(\xi,t) = -\left(c^2\xi^2 + \frac{3}{4}\right)B(\xi,t)
\end{align*}
and hence we get that
\begin{equation*}
    u_{tt} + u_{t} + u = -c^2\xi^2 u(x,t) = c^2u_{xx}
\end{equation*}
and clearly
\begin{equation*}
    u(x,0) = \frac{e^{-t/2}}{\sqrt{2\pi}} \int_{-\infty}^\infty \hat{f} e^{i\xi x} d\xi = f(x) 
\end{equation*}
and
\begin{equation*}
    u_t(x,0) = -\frac{f}{2} + \hat{g} + \frac{1}{2}\hat{f} = g(x)
\end{equation*}
thus our $u(x,t)$ is a solution to problem.

\section*{Problem 6.24}
Suppose we have the Klein-Gordon equation
\begin{equation*}
    u_{tt} - \Delta u + u = 0
\end{equation*}
for $x \in \R^d$ and $t > 0$ where also $u(x,0) = f(x)$ and $u_t(x,0) = g(x)$. Taking the fourier transform of both sides yields
\begin{equation*}
    \frac{\partial^2}{\partial t^2} \hat{u} + (|\xi|^2 +1)\hat{u} = 0
\end{equation*}
solving the above ODE yields
\begin{equation*}
    \hat{u}(x,t) = c_2 \sin\left(\sqrt{|\xi|^2 + 1}t\right) + c_1 \cos\left(\sqrt{|\xi|^2 + 1}t\right)
\end{equation*}
using the initial conditions gives us the particular solution
\begin{equation*}
    \hat{u}(x,t) = \hat{f}(x) \sin\left(\sqrt{|\xi|^2 + 1}t\right) + \frac{\hat{g}(x)}{\sqrt{|\xi|^2 + 1}}\cos\left(\sqrt{|\xi|^2 + 1}t\right)
\end{equation*}
then taking the fourier inverse yields
\begin{equation*}
    u(x,t) = \frac{1}{\sqrt{2\pi}} \int_{\infty}^\infty\left[\hat{f}(x) \sin\left(\sqrt{|\xi|^2 + 1}t\right) + \frac{\hat{g}(x)}{\sqrt{|\xi|^2 + 1}}\cos\left(\sqrt{|\xi|^2 + 1}t\right)\right]e^{ix\xi} \, d\xi 
\end{equation*}

\section*{Problem 6.26}
Consider $L^2((0,\infty))$ as a real Hilbert space. Let $A: L^2((0,\infty)) \to L^2((-\infty,\infty))$ be defined by
\begin{equation*}
    Au(x) = u(x) + \int_0^\infty \omega(x-y)u(y) \, dy
\end{equation*}
where $\omega \in L^1((0,\infty)) \cap L^2((0,\infty)) \cap C^2((0,\infty))$ is nonnegative, decreasing, convex, and even.

\subsection*{Part a}
We can extend $u \in L^2(0,\infty)$ to $\Tilde{u} \in L^2(-\infty,\infty)$ by defining $\Tilde{u}$ as
\begin{equation*}
    \Tilde{u}(x) = \begin{cases}
        u(x), & x \in (0,\infty) \\
        0, & x \not\in (0,\infty)
    \end{cases}
\end{equation*}
then we get that
\begin{equation*}
    Au(x) = \Tilde{u}(x) + \int_{-\infty}^\infty \omega(x-y) \title{u}(y) \, dy
\end{equation*}
additionally notice that
\begin{equation*}
    \int_{-\infty}^\infty \omega(x-y) \title{u}(y) \, dy = \omega * \Tilde{u}
\end{equation*}
and by Young's Generalized Inequality we have that $\omega * \Tilde{u} \in L^2(-\infty, \infty)$, and hence $Au \in L^2(-\infty,\infty)$.

\subsection*{Part b}
Let $u \in L^2(0,\infty)$ and $v \in L^2(-\infty,\infty)$, then we see that
\begin{align*} 
    \inner{Au}{v}_{L^2(0,\infty)} &= \int_{0}^\infty Au(x) v(x) \, dx \\
    &=\int_0^\infty \Tilde{u}(x)v(x) + \int_{-\infty}^\infty \omega(x-y)v(x) \Tilde{u}(y) \, dy \, dx \\
    &= \int_{-\infty}^\infty \Tilde{u}(x)v(x) + \int_{-\infty}^\infty \omega(x-y)v(x) \Tilde{u}(y) \, dy \, dx 
\end{align*}
and since $\omega(x-y) = \omega(y-x)$, we get that
\begin{align*}
\int_{-\infty}^\infty \Tilde{u}(x)v(x) + \int_{-\infty}^\infty \omega(x-y)v(x) \Tilde{u}(y) \, dy \, dx &= \int_{0}^\infty \Tilde{u}(y)v(y) + \Tilde{u(x)} \int_{0}^\infty \omega(y-x) v(x) \, dx \,dy \\
&= \int_{0}^\infty \left[v(y) + \int_{0}^\infty \omega(y-x) v(x) \, dx\right] \Tilde{u}(y) \, dy \\
&= \int_{0}^\infty Av(y) u(y) \, dy \\
&= \inner{u}{Av}_{L^2(0,\infty)}
\end{align*}
Thus $A$ is symmetric.

\subsection*{Part c}
Note that
\begin{equation*}
    \hat{\omega} = \frac{1}{\sqrt{2\pi}} \int_{-\infty}^\infty \omega(x) e^{-ix\xi} \, dx = \frac{2}{\sqrt{2\pi}} \int_{0}^\infty \omega(x) e^{-ix\xi} \, dx 
\end{equation*}
and since $\omega(x) e^{-ix\xi} \geq 0$ for all $x \in \R$, we get that
\begin{equation*}
     \hat{\omega} =  \frac{2}{\sqrt{2\pi}} \int_{0}^\infty \omega(x) e^{-ix\xi} \, dx \geq 0
\end{equation*}
thus $\hat{\omega} \geq 0$.

\subsection*{Part d}
Since the fourier transform preserves inner products we have that
\begin{equation*}
    \inner{Au}{u}_{L^2(0,\infty)} = \inner{\mathcal{F}(Au)}{\mathcal{F}(u)}_{L^2(0,\infty)}
\end{equation*}
and notice that
\begin{align*}
    \mathcal{F}(Au) = \hat{u} + \sqrt{2\pi} \hat{u} \hat{\omega}
\end{align*}
and thus
\begin{equation*}
    \inner{\mathcal{F}(Au)}{\mathcal{F}(u)}_{L^2(0,\infty)} = \int_{0}^\infty \left(\hat{u} + \sqrt{2\pi} \hat{u} \hat{\omega}\right) \overline{\hat{u}(x)} \, dx = \int_0^\infty |\hat{u}|^2 (1 + \sqrt{2\pi} \hat{\omega})\, dx \geq \norm{\hat{u}}_{L^2(0,\infty)}
\end{equation*}
which we see is only $0$, if $u \equiv 0$ and is strictly greater than otherwise.

\subsection*{Part e}
Notice that if for $f \in L^2(-\infty, \infty)$ we have 
\begin{equation*}
    Au = \Tilde{u}(x) + \omega * \Tilde{u} = f
\end{equation*}
then by taking the fourier transform to both sides, we get
\begin{align*}
    \hat{\Tilde{u}} + \sqrt{2\pi} \hat{w}\hat{\Tilde{u}} = \hat{f} \implies \hat{\Tilde{u}} = \frac{\hat{f}}{1 + \sqrt{2\pi} \hat{\omega}}
\end{align*}
taking the inverse fourier transform we get
\begin{equation*}
    \Tilde{u}(x) = \frac{1}{\sqrt{2\pi}} \int_{-\infty}^\infty \frac{\hat{f}}{1 + \sqrt{2\pi} \hat{\omega}}\, d\xi
\end{equation*}
\subsection*{Part f}
Suppose that there exists $u_1$ and $u_2$ such that
\begin{equation*}
    Au_1 = Au_2 = f
\end{equation*}
and $u_1 \neq u_2$, then we have 
\begin{equation*}
    A(u_1 - u_2) = 0 \implies \mathcal{F}(A(u_1 - u_2)) = 0
\end{equation*}
which gives us
\begin{equation*}
    \hat{u_1} - \hat{u_2} + \sqrt{2\pi}\hat{\omega}(\hat{u_1} - \hat{u_2}) = 0 \implies (\hat{u_1} - \hat{u_2})(1 + \sqrt{2\pi}\hat{\omega}) = 0
\end{equation*}
but since $1 + \sqrt{2\pi}\hat{\omega} \neq 0$ and thus $\hat{u}_1 - \hat{u}_2 = 0$ and since the fourier transform is unique we get that $u_1 = u_2$. Thus the solution is unique.
\end{document}