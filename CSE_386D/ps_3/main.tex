\documentclass[12pt]{report}
\usepackage{scribe,graphicx,graphics}
\usepackage{bbm}
\usepackage{yhmath}
\newcommand{\norm}[1]{\left|\left|#1\right|\right|}
\newcommand{\inner}[2]{\left\langle#1,#2\right\rangle}

\course{CSE 386D} 	
\coursetitle{Methods of Applied Mathematics II}	
\semester{Spring 2025}
\lecturer{} % Due Date: {\bf Mon, Oct 3 2016}}
\lecturetitle{Problem Set}
\lecturenumber{2}   
\lecturedate{}    
\usepackage{enumerate}
\newcommand{\remind}[1]{\textcolor{red}{\textbf{#1}}} %To remind me of unfinished work to fix later
\newcommand{\hide}[1]{} %To hide large blocks of code without using % symbols

\newcommand{\ep}{\varepsilon}
\newcommand{\vp}{\varphi}
\newcommand{\lam}{\lambda}
\newcommand{\Lam}{\Lambda}
%\newcommand{\abs}[1]{\ensuremath{\left\lvert#1\right\rvert}} % This clashes with the physics package
%\newcommand{\norm}[1]{\ensuremath{\left\lVert#1\right\rVert}} % This clashes with the physics package
\newcommand{\floor}[1]{\ensuremath{\left\lfloor#1\right\rfloor}}
\newcommand{\ceil}[1]{\ensuremath{\left\lceil#1\right\rceil}}
\newcommand{\A}{\mathbb{A}}
\newcommand{\B}{\mathbb{B}}
\newcommand{\C}{\mathbb{C}}
\newcommand{\D}{\mathbb{D}}
\newcommand{\E}{\mathbb{E}}
\newcommand{\F}{\mathbb{F}}
\newcommand{\K}{\mathbb{K}}
\newcommand{\N}{\mathbb{N}}
\newcommand{\Q}{\mathbb{Q}}
\newcommand{\R}{\mathbb{R}}
\newcommand{\T}{\mathbb{T}}
\newcommand{\X}{\mathbb{X}}
\newcommand{\Y}{\mathbb{Y}}
\newcommand{\Z}{\mathbb{Z}}
\newcommand{\As}{\mathcal{A}}
\newcommand{\Bs}{\mathcal{B}}
\newcommand{\Cs}{\mathcal{C}}
\newcommand{\Ds}{\mathcal{D}}
\newcommand{\Es}{\mathcal{E}}
\newcommand{\Fs}{\mathcal{F}}
\newcommand{\Gs}{\mathcal{G}}
\newcommand{\Hs}{\mathcal{H}}
\newcommand{\Is}{\mathcal{I}}
\newcommand{\Js}{\mathcal{J}}
\newcommand{\Ks}{\mathcal{K}}
\newcommand{\Ls}{\mathcal{L}}
\newcommand{\Ms}{\mathcal{M}}
\newcommand{\Ns}{\mathcal{N}}
\newcommand{\Os}{\mathcal{O}}
\newcommand{\Ps}{\mathcal{P}}
\newcommand{\Qs}{\mathcal{Q}}
\newcommand{\Rs}{\mathcal{R}}
\newcommand{\Ss}{\mathcal{S}}
\newcommand{\Ts}{\mathcal{T}}
\newcommand{\Us}{\mathcal{U}}
\newcommand{\Vs}{\mathcal{V}}
\newcommand{\Ws}{\mathcal{W}}
\newcommand{\Xs}{\mathcal{X}}
\newcommand{\Ys}{\mathcal{Y}}
\newcommand{\Zs}{\mathcal{Z}}
\newcommand{\ab}{\textbf{a}}
\newcommand{\bb}{\textbf{b}}
\newcommand{\cb}{\textbf{c}}
\newcommand{\db}{\textbf{d}}
\newcommand{\ub}{\textbf{u}}
\newcommand{\sbb}{\textbf{s}}
%\renewcommand{\vb}{\textbf{v}} % This clashes with the physics package (the physics package already defines the \vb command)
\newcommand{\wb}{\textbf{w}}
\newcommand{\xb}{\textbf{x}}
\newcommand{\yb}{\textbf{y}}
\newcommand{\zb}{\textbf{z}}
\newcommand{\vbb}{\textbf{v}}
\newcommand{\Ab}{\textbf{A}}
\newcommand{\Bb}{\textbf{B}}
\newcommand{\Cb}{\textbf{C}}
\newcommand{\Db}{\textbf{D}}
\newcommand{\eb}{\textbf{e}}
\newcommand{\ex}{\textbf{e}_x}
\newcommand{\ey}{\textbf{e}_y}
\newcommand{\ez}{\textbf{e}_z}
\newcommand{\zerob}{\mathbf{0}}
\newcommand{\abar}{\overline{a}}
\newcommand{\bbar}{\overline{b}}
\newcommand{\cbar}{\overline{c}}
\newcommand{\dbar}{\overline{d}}
\newcommand{\ubar}{\overline{u}}
\newcommand{\vbar}{\overline{v}}
\newcommand{\wbar}{\overline{w}}
\newcommand{\xbar}{\overline{x}}
\newcommand{\ybar}{\overline{y}}
\newcommand{\zbar}{\overline{z}}
\newcommand{\Abar}{\overline{A}}
\newcommand{\Bbar}{\overline{B}}
\newcommand{\Cbar}{\overline{C}}
\newcommand{\Dbar}{\overline{D}}
\newcommand{\Ubar}{\overline{U}}
\newcommand{\Vbar}{\overline{V}}
\newcommand{\Wbar}{\overline{W}}
\newcommand{\Xbar}{\overline{X}}
\newcommand{\Ybar}{\overline{Y}}
\newcommand{\Zbar}{\overline{Z}}
\newcommand{\Aint}{A^\circ}
\newcommand{\Bint}{B^\circ}
\newcommand{\limk}{\lim_{k\to\infty}}
\newcommand{\limm}{\lim_{m\to\infty}}
\newcommand{\limn}{\lim_{n\to\infty}}
\newcommand{\limx}[1][a]{\lim_{x\to#1}}
\newcommand{\liminfm}{\liminf_{m\to\infty}}
\newcommand{\limsupm}{\limsup_{m\to\infty}}
\newcommand{\liminfn}{\liminf_{n\to\infty}}
\newcommand{\limsupn}{\limsup_{n\to\infty}}
\newcommand{\sumkn}{\sum_{k=1}^n}
\newcommand{\sumk}[1][1]{\sum_{k=#1}^\infty}
\newcommand{\summ}[1][1]{\sum_{m=#1}^\infty}
\newcommand{\sumn}[1][1]{\sum_{n=#1}^\infty}
\newcommand{\emp}{\varnothing}
\newcommand{\exc}{\backslash}
\newcommand{\sub}{\subseteq}
\newcommand{\sups}{\supseteq}
\newcommand{\capp}{\bigcap}
\newcommand{\cupp}{\bigcup}
\newcommand{\kupp}{\bigsqcup}
\newcommand{\cappkn}{\bigcap_{k=1}^n}
\newcommand{\cuppkn}{\bigcup_{k=1}^n}
\newcommand{\kuppkn}{\bigsqcup_{k=1}^n}
\newcommand{\cappk}[1][1]{\bigcap_{k=#1}^\infty}
\newcommand{\cuppk}[1][1]{\bigcup_{k=#1}^\infty}
\newcommand{\cappm}[1][1]{\bigcap_{m=#1}^\infty}
\newcommand{\cuppm}[1][1]{\bigcup_{m=#1}^\infty}
\newcommand{\cappn}[1][1]{\bigcap_{n=#1}^\infty}
\newcommand{\cuppn}[1][1]{\bigcup_{n=#1}^\infty}
\newcommand{\kuppk}[1][1]{\bigsqcup_{k=#1}^\infty}
\newcommand{\kuppm}[1][1]{\bigsqcup_{m=#1}^\infty}
\newcommand{\kuppn}[1][1]{\bigsqcup_{n=#1}^\infty}
\newcommand{\cappa}{\bigcap_{\alpha\in I}}
\newcommand{\cuppa}{\bigcup_{\alpha\in I}}
\newcommand{\kuppa}{\bigsqcup_{\alpha\in I}}
\newcommand{\Rx}{\overline{\mathbb{R}}}
\newcommand{\dx}{\,dx}
\newcommand{\dy}{\,dy}
\newcommand{\dt}{\,dt}
\newcommand{\dax}{\,d\alpha(x)}
\newcommand{\dbx}{\,d\beta(x)}
\DeclareMathOperator{\glb}{\text{glb}}
\DeclareMathOperator{\lub}{\text{lub}}
\newcommand{\xh}{\widehat{x}}
\newcommand{\yh}{\widehat{y}}
\newcommand{\zh}{\widehat{z}}
\newcommand{\<}{\langle}
\renewcommand{\>}{\rangle}
\renewcommand{\iff}{\Leftrightarrow}
\DeclareMathOperator{\im}{\text{im}}
\let\spn\relax\let\Re\relax\let\Im\relax
\DeclareMathOperator{\spn}{\text{span}}
\DeclareMathOperator{\sym}{\text{Sym}}
\DeclareMathOperator{\myskew}{\text{Skew}}
\DeclareMathOperator{\Re}{\text{Re}}
\DeclareMathOperator{\Im}{\text{Im}}
\DeclareMathOperator{\diag}{\text{diag}}
\endinput

% Insert your name here!
\scribe{Student Name: Noah Reef}

\begin{document}
\maketitle

\section*{Problem 6.6}
Consider the function $f(x) = \frac{\sin(|x|)}{|x|}$, then we see that
\begin{equation*}
    \int_{\R^d} \left|\frac{\sin(|x|)}{|x|}\right|^2 \, dx  = \pi < \infty
\end{equation*}
but 
\begin{equation*}
    \int_{\R^d} \left|\frac{\sin(|x|)}{|x|}\right| \, dx = \infty
\end{equation*}
thus $f \in L^2(\R^d)$ but $f \not\in L^1(\R^d)$, and we see that
\begin{align*}
    \int_{\R^d} \left|\hat{f}(\xi)\right| \, d\xi &= \int_{\R^d} |\mathbbm{1}_1| \,dx = 2 < \infty
\end{align*}
thus $\hat{f} \in L^1(\R^d)$. This can happen for $L^2(\R^d)$ functions who are small but do not vanish, be in $L^1(\R^d)$, so that once under fourier transform do vanish towards infinity.

\section*{Problem 6.7}
Let $f \in L^p(\R^d)$ for $1 \leq p \leq 2$.
\subsection*{Part a}
Let $A = \{|f| > 1\}$ and define $f_1 := f\chi_A$  and $f_2:= f - f_1$, then clearly $f = f_1 + f_2$, and we have that 
\begin{equation*}
    \norm{f_1}_1 = \int_{\R^d} |f(x) \cdot \chi_A(x)| \, dx = \int_A |f(x)| \, dx \leq \int_A |f(x)|^p \, dx = \norm{f}_p^p < \infty
\end{equation*}
and, noticing that $f_2 = f\chi_{\R^d \setminus A}$, we have 
\begin{equation*}
    \norm{f_2}_2^2 = \int_{\R^d} |f(x) \cdot \chi_{\R^d \setminus A}(x)|^2 \, dx = \int_{\R^d \setminus A} |f(x)|^2 \, dx \leq \int_{\R^d \setminus A} |f(x)|^p \, dx = \norm{f}_p^p < \infty
\end{equation*}
and thus we see that $f_1 \in L^1(\R^d)$ and $f_2 \in L^2(\R^d)$.
\subsection*{Part b}
Define $\hat{f} = \hat{f}_1 + \hat{f}_2$, and suppose there exists $f_1,g_1 \in L^1(\R^d)$ and $f_2,g_2 \in L^2(\R^d)$ such that
\begin{equation*}
    \hat{f} = \hat{f}_1 + \hat{f}_2 = \hat{g}_1 + \hat{g}_2
\end{equation*}
then we have that
\begin{align*}
    \widehat{f_1 - g_1} - \widehat{g_2 - f_2} = 0
\end{align*}
and since the fourier transform is injective we have that
\begin{equation*}
    f_1 - g_1 = g_2 - f_2
\end{equation*}
which implies that $f_1 = g_1$ and $f_2 = g_2$

\section*{Problem 6.8}
Let the field be complex and define $T: L^2(\R^d) \to L^2(\R^d)$ by
\begin{equation*}
    Tf(x) = \int e^{-|x-y|^2/2}f(y) \, dy 
\end{equation*}
We can see that by Plancherel's Theorem that
\begin{equation*}
    \inner{Tf}{f}_{L^2} = \int Tf(x) \overline{f(x)}\, dx = \int \widehat{Tf}(\xi) \overline{\hat{f}(\xi)} \, d\xi
\end{equation*}
let $h(x) = e^{-|x|^2/2}$ then we see that $Tf(x) = (f*h)(x)$ and hence
\begin{equation*}
    \widehat{Tf} = (2\pi)^{d/2} \hat{f}\hat{h}
\end{equation*}
and from exercise $6.2$, we have that
\begin{equation*}
    \hat{h}(\xi) = e^{-|\xi|^2/2}
\end{equation*}
thus 
\begin{equation*}
    \int \widehat{Tf}(\xi) \overline{\hat{f}(\xi)} \, d\xi = (2\pi)^{d/2} \int e^{-|\xi|^2/2} |\hat{f}(\xi)|^2 \, dx \geq 0
\end{equation*}
and we note that equality occurs only when $f = 0$. Thus $T$ is a positive operator. Next suppose there exists $f,g \in L^2(\R^d)$ such that $Tf = Tg$ but $g \neq f$. Then we see that 
\begin{align*}
    \widehat{(f * h)} &= \widehat{(f*g)} \\
    (2\pi)^{d/2}\hat{h} \hat{f} &=(2\pi)^{d/2}\hat{h} \hat{g} \\
    \hat{f} &= \hat{g}
\end{align*}
and since $\mathcal{F}: L^2(\R^d) \to L^2(\R^d)$ is one-to-one and onto that we have $f = g$, which contradicts our original assumption that $f \neq g$ and hence $T$ is injective. Additionally notice that if we have $g(x) = (2\pi)^{d/2} h$ then $\hat{g}(\xi) = (2\pi)^{d/2} h$ and hence for $T$ to be onto we need there to exist $f$ such that
\begin{equation*}
    (2\pi)^{d/2}h \hat{f} = (2\pi)^{d/2}h \implies f = 1
\end{equation*}
however $1 \not\in L^2(\R^d)$, thus $T$ is not surjective.

\section*{Problem 6.14}
Let $\varphi \in \mathcal{S}(\R^d)$, $\hat{\varphi}(0) = (2\pi)^{-d/2}$ and $\varphi_\epsilon = (1/\epsilon^d)\varphi\left(\frac{x}{\epsilon}\right)$. Notice that since $\hat{\varphi}(0) = (2\pi)^{-d/2}$, we have that
\begin{equation*}
    \hat{\varphi}(0) = (2\pi)^{-d/2}\int \varphi(x) \, dx = (2\pi)^{-d/2}
\end{equation*}
which implies that
\begin{equation*}
    \int \varphi(x) \, dx = 1
\end{equation*}
then for any $\phi \in \mathcal{S}$, we have that
\begin{equation*}
    \inner{\varphi_\epsilon}{\phi} = \int \varphi_\epsilon(x) \phi(x) \, dx = \int \phi(x) \frac{1}{\epsilon^d} \varphi(x/\epsilon) \, dx = \int \phi(\epsilon y) \varphi(y)\, dy
\end{equation*}
then by the Dominated Convergence Theorem, we have that as $\epsilon \to 0$ that
\begin{equation*}
    \int \phi(\epsilon y) \varphi(y)\, dy \to \phi(0) \int \varphi(y) \, dy = \phi(0) = \inner{\delta_0}{\phi}
\end{equation*}
thus 
\begin{equation*}
    \varphi_\epsilon \xrightarrow{\mathcal{S}'} \delta_0
\end{equation*}
and since the $\mathcal{F}: \mathcal{S}' \to \mathcal{S}'$ is continuous we have that
\begin{equation*}
    \hat{\varphi}_\epsilon \to \hat{\delta}_0 = (2\pi)^{-d/2}
\end{equation*}
Note that these converge in the sense of tempered distributions.

\section*{Problem 6.15}
If we consider the function with height $f(x) = e^{x^2}$ centered at $x = 1,2,3,\dots$ with width $1/x!$, and $0$ otherwise. We clearly see that there exists $x \in \R^d$ such that $|f(x)| > P(x)$ for any polynomial $P$ and $f \in L^1(\R^d)$. Then we have that for any $\phi \in \mathcal{S}$, we have that
\begin{equation*}
    \left| \int \phi(x)f(x) \, dx\right| \leq \norm{f}_{L^1}\norm{\phi}_{L^\infty} \leq C
\end{equation*}
thus by Theorem 6.31, we have that $f$ is a distribution.
\end{document}