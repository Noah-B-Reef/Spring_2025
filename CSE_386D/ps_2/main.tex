\documentclass[12pt]{report}
\usepackage{scribe,graphicx,graphics}
\usepackage{bbm}
\usepackage{yhmath}
\newcommand{\norm}[1]{\left|\left|#1\right|\right|}

\course{CSE 386D} 	
\coursetitle{Methods of Applied Mathematics II}	
\semester{Spring 2025}
\lecturer{} % Due Date: {\bf Mon, Oct 3 2016}}
\lecturetitle{Problem Set}
\lecturenumber{2}   
\lecturedate{}    
\input{commands.tex}

% Insert your name here!
\scribe{Student Name: Noah Reef}

\begin{document}
\maketitle

\section*{Problem 6.6}
Consider the function $f(x) = \frac{\sin(|x|)}{|x|}$, then we see that
\begin{equation*}
    \int_{\R^d} \left|\frac{\sin(|x|)}{|x|}\right|^2 \, dx  = \pi < \infty
\end{equation*}
but 
\begin{equation*}
    \int_{\R^d} \left|\frac{\sin(|x|)}{|x|}\right| \, dx = \infty
\end{equation*}
thus $f \in L^2(\R^d)$ but $f \not\in L^1(\R^d)$, and we see that
\begin{align*}
    \int_{\R^d} \left|\hat{f}(\xi)\right| \, d\xi &= \int_{\R^d} |\mathbbm{1}_1| \,dx = 2 < \infty
\end{align*}
thus $\hat{f} \in L^1(\R^d)$. [INSERT THE "UNDER WHAT CIRCUMSTANCES THIS OCCURS"...]

\section*{Problem 6.7}
Let $f \in L^p(\R^d)$ for $1 \leq p \leq 2$.
\subsection*{Part a}
Let $A = \{|f| > 1\}$ and define $f_1 := f\chi_A$  and $f_2:= f - f_1$, then clearly $f = f_1 + f_2$, and we have that 
\begin{equation*}
    \norm{f_1}_1 = \int_{\R^d} |f(x) \cdot \chi_A(x)| \, dx = \int_A |f(x)| \, dx \leq \int_A |f(x)|^p \, dx = \norm{f}_p^p < \infty
\end{equation*}
and, noticing that $f_2 = f\chi_{\R^d \setminus A}$, we have 
\begin{equation*}
    \norm{f_2}_2^2 = \int_{\R^d} |f(x) \cdot \chi_{\R^d \setminus A}(x)|^2 \, dx = \int_{\R^d \setminus A} |f(x)|^2 \, dx \leq \int_{\R^d \setminus A} |f(x)|^p \, dx = \norm{f}_p^p < \infty
\end{equation*}
and thus we see that $f_1 \in L^1(\R^d)$ and $f_2 \in L^2(\R^d)$.
\subsection*{Part b}
Define $\hat{f} = \hat{f}_1 + \hat{f}_2$, then we see that
\begin{align*}
    \hat{f} &= \widehat{(f_1 + f_2)} \\
            &= (2\pi)^{-d/2} \int_{\R^d} (f_1(x) + f_2(x)) e^{-ix \cdot \xi}\, dx \\
\end{align*}
which (JUSTIFY)
\begin{align*}
    (2\pi)^{-d/2} \int_{\R^d} (f_1(x) + f_2(x)) e^{-ix \cdot \xi}\, dx &= (2\pi)^{-d/2} \int_{\R^d} f_1(x) e^{-ix \cdot \xi}\, dx + (2\pi)^{-d/2} \int_{\R^d} f_2(x) e^{-ix \cdot \xi}\, dx \\
    &= \hat{f}_1 + \hat{f}_2
\end{align*}
thus the definition above is independent of the choice of $f_1$ and $f_2$, thus is well-defined.

\section*{Problem 6.8}

\end{document}