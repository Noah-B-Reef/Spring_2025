\documentclass{article}

\usepackage{amsmath, amsthm, amssymb, amsfonts}
\usepackage{thmtools}
\usepackage{graphicx}
\usepackage{setspace}
\usepackage{geometry}
\usepackage{float}
\usepackage{hyperref}
\usepackage[utf8]{inputenc}
\usepackage[english]{babel}
\usepackage{framed}
\usepackage[dvipsnames]{xcolor}
\usepackage{tcolorbox}
\newcommand{\norm}[1]{\left\lVert#1\right\rVert}

\colorlet{LightGray}{White!90!Periwinkle}
\colorlet{LightOrange}{Orange!15}
\colorlet{LightGreen}{Green!15}

\newcommand{\RR}{\mathbb{R}}
\newcommand{\CC}{\mathbb{C}}
\newcommand{\ZZ}{\mathbb{Z}}
\newcommand{\NN}{\mathbb{N}}
\newcommand{\QQ}{\mathbb{Q}}
\newcommand{\FT}{\mathcal{F}}

\newcommand{\HRule}[1]{\rule{\linewidth}{#1}}

\declaretheoremstyle[name=Theorem,]{thmsty}
\declaretheorem[style=thmsty,numberwithin=section]{theorem}
\tcolorboxenvironment{theorem}{colback=LightGray}

\declaretheoremstyle[name=Proposition,]{prosty}
\declaretheorem[style=prosty,numberlike=theorem]{proposition}
\tcolorboxenvironment{proposition}{colback=LightOrange}

\declaretheoremstyle[name=Principle,]{prcpsty}
\declaretheorem[style=prcpsty,numberlike=theorem]{principle}
\tcolorboxenvironment{principle}{colback=LightGreen}

\setstretch{1.2}
\geometry{
    textheight=9in,
    textwidth=5.5in,
    top=1in,
    headheight=12pt,
    headsep=25pt,
    footskip=30pt
}

% ------------------------------------------------------------------------------

\begin{document}
% ------------------------------------------------------------------------------
% Cover Page and ToC
% ------------------------------------------------------------------------------

\title{ \normalsize \textsc{}
		\\ [2.0cm]
		\HRule{1.5pt} \\
		\LARGE \textbf{\uppercase{CSE 386D Notes}
		\HRule{2.0pt} \\ [0.6cm] \LARGE{Spring 2025} \vspace*{10\baselineskip}}
		}
\date{}
\author{\textbf{Author} \\ 
		Noah Reef \\
	  UT Austin	\\
		Spring 2025}

\maketitle
\newpage

\tableofcontents
\newpage

% ------------------------------------------------------------------------------

\section{The Fourier Transform}
\subsection{The $L^1(\mathbb{R}^d)$ Theory}

If $\xi \in \mathbb{R}^d$, the function
\begin{equation*}
  \varphi_\xi(x) = e^{-ix\cdot\xi} =  \cos(x \cdot \xi) - i \sin(x \cdot \xi)
\end{equation*}
for $x \in \mathbb{R}^d$ is a plane wave in the direction $\xi$. Its period in the $j$th direction is $1\pi/\xi_j$.

\begin{proposition}
  For such $\varphi$ we have the following:
\begin{enumerate}
  \item $|\varphi_\xi| = 1$ and $\bar{\varphi_\xi} = \varphi_{-\xi}$ for any $\xi \in \mathbb{R}^d$ 
  \item $\varphi_\xi(x + y) = \varphi_\xi(x)\varphi_\xi(y)$ for any $x,y,\xi \in \mathbb{R}^d$ 
  \item $-\Delta \varphi_\xi = |\xi|^2 \varphi_\xi$ for any $\xi \in \mathbb{R}^d$
\end{enumerate}
\end{proposition}
\begin{principle}
  If $f \in L^1(\mathbb{R}^d)$, the Fourier transform of $f$ is 
  \begin{equation*}
    \mathcal{F}f(\xi) = \hat{f}(\xi) = (2\pi)^{-d/2} \int_{\mathbb{R}^d} f(x)e^{-ix \cdot \xi} \, dx
  \end{equation*}
\end{principle}

\begin{proposition}
  The Fourier transform
  \begin{equation*}
    \mathcal{F}: L^1(\mathbb{R}^d) \to L^\infty (\mathcal{R}^d)
  \end{equation*}
  is a bounded linear operator, and 
  \begin{equation*}
    \norm{\hat{f}}_{L^\infty (\mathcal{R}^d)} \leq (2\pi)^{-d/2} \norm{f}_{L^1(\mathbb{R}^d)}
      
  \end{equation*}
\end{proposition}

\begin{proposition}
  If $f \in L^1(\mathbb{R}^d)$ and $\tau_y$ is a translation by $y$, then
  \begin{enumerate}
  \item $\mathcal{F}(\tau_yf)(\xi) = e^{-iy \cdot \xi}\hat{f}(\xi)$ for all $y \in \mathbb{R}^d$. 
\item $\mathcal{F}(e^{ix \cdot y}f)(\xi) = \tau_y \hat{f}(\xi)$ for all $y \in \mathbb{R}^d$ 
    \item if $r > 0$ is given,
      \begin{equation*}
        \mathcal{F}(f(rx))(\xi) = r^{-d}\hat{f}(r^{-1}\xi)
      \end{equation*}
  
    \item $\hat{\bar{f}}(\xi) = \overline{\hat{f}(-\xi)}$
  \end{enumerate}
\end{proposition}

\begin{principle}
  A continuous function $f$ on $\mathbb{R}^d$ is said to vanish at infinity if for any $\epsilon > 0$ there is $K \subset\subset \mathbb{R}^d$ such that
  \begin{equation*}
    |f(x)| < \epsilon
  \end{equation*}
  for $x \not\in K$, The subspace of all such continuous functions is denoted
  \begin{equation*}
    C_v(\mathbb{R}^d) = \{f \in C^0(\mathbb{R}^d): \text{$f$ vanishes at $\infty$}\}
  \end{equation*}
\end{principle}

\begin{theorem}
  The space $C_v(\mathbb{R}^d)$ is a closed linear subspace of $L^\infty(\mathbb{R}^d)$
\end{theorem}

\begin{theorem}[Riemann-Lebesgue Lemma]
  The Fourier transform
  \begin{equation*}
    \mathcal{F}: L^1(\mathbb{R}^d) \to C_v(\mathbb{R}^d) \subset L^\infty(\mathbb{R}^d)
  \end{equation*}
  Then for $f \in L^1(\mathbb{R}^d)$
  \begin{equation*}
    \lim_{|\xi| \to \infty}|\hat{f}(\xi)| = 0 \quad \text{ and } \quad \hat{f} \in C^0(\mathbb{R}^d)
  \end{equation*}
\end{theorem}
\begin{proposition}
  If $f,g \in L^1(\mathbb{R}^d)$, then
  \begin{enumerate}
    \item $\int \mathcal{F}(f)g = \int f\mathcal{F}(g)$ 
  \item $f * g \in L^1(\mathbb{R}^d)$ and $\mathcal{F}(f * g) = (2\pi)^{d/2}\mathcal{F}(f)\mathcal{F}(g)$
  \end{enumerate}
\end{proposition}
\begin{theorem}[Generalized Young's Inequality]
  Suppose $K(x,y)$ is measurable of $\RR^d \times \RR^d$ and there is some $C > 0$ such that
  \begin{equation*}
    \int |K(x,y)| \, dx \leq C \quad \text{ and } \quad \int |K(x,y)| \, dy \leq C
  \end{equation*}
  for almost every $x,y \in \RR^d$, respectively. Define the operator $T$ by
  \begin{equation*}
    Tf(x) = \int K(x,y) f(y) \, dy
  \end{equation*}
  If $1 \leq p \leq \infty$, then $T: L^p(\RR^d) \to L^p(\RR^d)$ is a bounded linear operator with operator norm $\norm{T} \leq C$.
\end{theorem}
\begin{proposition}[Young's Inequality]
  If $1 \leq p \leq \infty$, $f \in L^p(\RR^d)$ and $g \in L^1(\RR^d)$, then $f * g \in L^p(\RR^d)$ and
  \begin{equation*}
    \norm{f * g}_p \leq \norm{f}_p \norm{g}_1
  \end{equation*}
\end{proposition}

\begin{theorem}[Paley-Wiener Theorem]
  If $f \in C_0^\infty(\RR^d)$, then $\FT(f)$ extend to an entire holomorphic function on $\CC^d$.
\end{theorem}
\newpage

% ------------------------------------------------------------------------------
% Reference and Cited Works
% ------------------------------------------------------------------------------

\bibliographystyle{IEEEtran}
\bibliography{References.bib}

% ------------------------------------------------------------------------------

\end{document}
