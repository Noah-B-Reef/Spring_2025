\documentclass[12pt]{report}
\usepackage{scribe,graphicx,graphics}
\usepackage{bbm}
\usepackage{yhmath}
\usepackage{stackengine}
\newcommand\dhookrightarrow{\mathrel{%
  \ensurestackMath{\stackanchor[.1ex]{\hookrightarrow}{\hookrightarrow}}
}}
\newcommand{\norm}[1]{\left|\left|#1\right|\right|}
\newcommand{\inner}[2]{\left\langle#1,#2\right\rangle}

\course{CSE 386D} 	
\coursetitle{Methods of Applied Mathematics II}	
\semester{Spring 2025}
\lecturer{} % Due Date: {\bf Mon, Oct 3 2016}}
\lecturetitle{Problem Set}
\lecturenumber{8}   
\lecturedate{}    
\usepackage{enumerate}
\newcommand{\remind}[1]{\textcolor{red}{\textbf{#1}}} %To remind me of unfinished work to fix later
\newcommand{\hide}[1]{} %To hide large blocks of code without using % symbols

\newcommand{\ep}{\varepsilon}
\newcommand{\vp}{\varphi}
\newcommand{\lam}{\lambda}
\newcommand{\Lam}{\Lambda}
%\newcommand{\abs}[1]{\ensuremath{\left\lvert#1\right\rvert}} % This clashes with the physics package
%\newcommand{\norm}[1]{\ensuremath{\left\lVert#1\right\rVert}} % This clashes with the physics package
\newcommand{\floor}[1]{\ensuremath{\left\lfloor#1\right\rfloor}}
\newcommand{\ceil}[1]{\ensuremath{\left\lceil#1\right\rceil}}
\newcommand{\A}{\mathbb{A}}
\newcommand{\B}{\mathbb{B}}
\newcommand{\C}{\mathbb{C}}
\newcommand{\D}{\mathbb{D}}
\newcommand{\E}{\mathbb{E}}
\newcommand{\F}{\mathbb{F}}
\newcommand{\K}{\mathbb{K}}
\newcommand{\N}{\mathbb{N}}
\newcommand{\Q}{\mathbb{Q}}
\newcommand{\R}{\mathbb{R}}
\newcommand{\T}{\mathbb{T}}
\newcommand{\X}{\mathbb{X}}
\newcommand{\Y}{\mathbb{Y}}
\newcommand{\Z}{\mathbb{Z}}
\newcommand{\As}{\mathcal{A}}
\newcommand{\Bs}{\mathcal{B}}
\newcommand{\Cs}{\mathcal{C}}
\newcommand{\Ds}{\mathcal{D}}
\newcommand{\Es}{\mathcal{E}}
\newcommand{\Fs}{\mathcal{F}}
\newcommand{\Gs}{\mathcal{G}}
\newcommand{\Hs}{\mathcal{H}}
\newcommand{\Is}{\mathcal{I}}
\newcommand{\Js}{\mathcal{J}}
\newcommand{\Ks}{\mathcal{K}}
\newcommand{\Ls}{\mathcal{L}}
\newcommand{\Ms}{\mathcal{M}}
\newcommand{\Ns}{\mathcal{N}}
\newcommand{\Os}{\mathcal{O}}
\newcommand{\Ps}{\mathcal{P}}
\newcommand{\Qs}{\mathcal{Q}}
\newcommand{\Rs}{\mathcal{R}}
\newcommand{\Ss}{\mathcal{S}}
\newcommand{\Ts}{\mathcal{T}}
\newcommand{\Us}{\mathcal{U}}
\newcommand{\Vs}{\mathcal{V}}
\newcommand{\Ws}{\mathcal{W}}
\newcommand{\Xs}{\mathcal{X}}
\newcommand{\Ys}{\mathcal{Y}}
\newcommand{\Zs}{\mathcal{Z}}
\newcommand{\ab}{\textbf{a}}
\newcommand{\bb}{\textbf{b}}
\newcommand{\cb}{\textbf{c}}
\newcommand{\db}{\textbf{d}}
\newcommand{\ub}{\textbf{u}}
\newcommand{\sbb}{\textbf{s}}
%\renewcommand{\vb}{\textbf{v}} % This clashes with the physics package (the physics package already defines the \vb command)
\newcommand{\wb}{\textbf{w}}
\newcommand{\xb}{\textbf{x}}
\newcommand{\yb}{\textbf{y}}
\newcommand{\zb}{\textbf{z}}
\newcommand{\vbb}{\textbf{v}}
\newcommand{\Ab}{\textbf{A}}
\newcommand{\Bb}{\textbf{B}}
\newcommand{\Cb}{\textbf{C}}
\newcommand{\Db}{\textbf{D}}
\newcommand{\eb}{\textbf{e}}
\newcommand{\ex}{\textbf{e}_x}
\newcommand{\ey}{\textbf{e}_y}
\newcommand{\ez}{\textbf{e}_z}
\newcommand{\zerob}{\mathbf{0}}
\newcommand{\abar}{\overline{a}}
\newcommand{\bbar}{\overline{b}}
\newcommand{\cbar}{\overline{c}}
\newcommand{\dbar}{\overline{d}}
\newcommand{\ubar}{\overline{u}}
\newcommand{\vbar}{\overline{v}}
\newcommand{\wbar}{\overline{w}}
\newcommand{\xbar}{\overline{x}}
\newcommand{\ybar}{\overline{y}}
\newcommand{\zbar}{\overline{z}}
\newcommand{\Abar}{\overline{A}}
\newcommand{\Bbar}{\overline{B}}
\newcommand{\Cbar}{\overline{C}}
\newcommand{\Dbar}{\overline{D}}
\newcommand{\Ubar}{\overline{U}}
\newcommand{\Vbar}{\overline{V}}
\newcommand{\Wbar}{\overline{W}}
\newcommand{\Xbar}{\overline{X}}
\newcommand{\Ybar}{\overline{Y}}
\newcommand{\Zbar}{\overline{Z}}
\newcommand{\Aint}{A^\circ}
\newcommand{\Bint}{B^\circ}
\newcommand{\limk}{\lim_{k\to\infty}}
\newcommand{\limm}{\lim_{m\to\infty}}
\newcommand{\limn}{\lim_{n\to\infty}}
\newcommand{\limx}[1][a]{\lim_{x\to#1}}
\newcommand{\liminfm}{\liminf_{m\to\infty}}
\newcommand{\limsupm}{\limsup_{m\to\infty}}
\newcommand{\liminfn}{\liminf_{n\to\infty}}
\newcommand{\limsupn}{\limsup_{n\to\infty}}
\newcommand{\sumkn}{\sum_{k=1}^n}
\newcommand{\sumk}[1][1]{\sum_{k=#1}^\infty}
\newcommand{\summ}[1][1]{\sum_{m=#1}^\infty}
\newcommand{\sumn}[1][1]{\sum_{n=#1}^\infty}
\newcommand{\emp}{\varnothing}
\newcommand{\exc}{\backslash}
\newcommand{\sub}{\subseteq}
\newcommand{\sups}{\supseteq}
\newcommand{\capp}{\bigcap}
\newcommand{\cupp}{\bigcup}
\newcommand{\kupp}{\bigsqcup}
\newcommand{\cappkn}{\bigcap_{k=1}^n}
\newcommand{\cuppkn}{\bigcup_{k=1}^n}
\newcommand{\kuppkn}{\bigsqcup_{k=1}^n}
\newcommand{\cappk}[1][1]{\bigcap_{k=#1}^\infty}
\newcommand{\cuppk}[1][1]{\bigcup_{k=#1}^\infty}
\newcommand{\cappm}[1][1]{\bigcap_{m=#1}^\infty}
\newcommand{\cuppm}[1][1]{\bigcup_{m=#1}^\infty}
\newcommand{\cappn}[1][1]{\bigcap_{n=#1}^\infty}
\newcommand{\cuppn}[1][1]{\bigcup_{n=#1}^\infty}
\newcommand{\kuppk}[1][1]{\bigsqcup_{k=#1}^\infty}
\newcommand{\kuppm}[1][1]{\bigsqcup_{m=#1}^\infty}
\newcommand{\kuppn}[1][1]{\bigsqcup_{n=#1}^\infty}
\newcommand{\cappa}{\bigcap_{\alpha\in I}}
\newcommand{\cuppa}{\bigcup_{\alpha\in I}}
\newcommand{\kuppa}{\bigsqcup_{\alpha\in I}}
\newcommand{\Rx}{\overline{\mathbb{R}}}
\newcommand{\dx}{\,dx}
\newcommand{\dy}{\,dy}
\newcommand{\dt}{\,dt}
\newcommand{\dax}{\,d\alpha(x)}
\newcommand{\dbx}{\,d\beta(x)}
\DeclareMathOperator{\glb}{\text{glb}}
\DeclareMathOperator{\lub}{\text{lub}}
\newcommand{\xh}{\widehat{x}}
\newcommand{\yh}{\widehat{y}}
\newcommand{\zh}{\widehat{z}}
\newcommand{\<}{\langle}
\renewcommand{\>}{\rangle}
\renewcommand{\iff}{\Leftrightarrow}
\DeclareMathOperator{\im}{\text{im}}
\let\spn\relax\let\Re\relax\let\Im\relax
\DeclareMathOperator{\spn}{\text{span}}
\DeclareMathOperator{\sym}{\text{Sym}}
\DeclareMathOperator{\myskew}{\text{Skew}}
\DeclareMathOperator{\Re}{\text{Re}}
\DeclareMathOperator{\Im}{\text{Im}}
\DeclareMathOperator{\diag}{\text{diag}}
\endinput
\newcommand{\supp}[]{\text{supp}}

% Insert yousupp}
\scribe{Student Name: Noah Reef}

\begin{document}
\maketitle

\section*{Problem 8.1}
Suppose that $A$ is a positive definite matrix, then we have that $x^TAx \geq  0$ for all $x \in \R^d$ and has equality only when $x = 0$. Then we see that if $v$ is a normalized eigenvector of $A$ with corresponding eigenvalue $\lambda$ then we have that
\begin{equation*}
  v^TAv = v^T \lambda v = \lambda \norm{v}_2^2 = \lambda \geq 0
\end{equation*}
and thus the eigenvalues of $A$ are positive. Now suppose $A$ is symmetric matrix and has positive eigenvalues, then we have that $A$ is full rank and thus for all $x$ we have that
\begin{equation*}
 x = \sum_{j=1}^n \inner{x}{v_j}v_j 
\end{equation*}
where $v_j$ are the eigenvectors of $A$. Then we have that
\begin{equation*}
  x^TAx = \sum_{j=1}^n \inner{x}{v_j}v_j^TA\sum_{k=1}^n \inner{x}{v_k}v_k = \sum_{j=1}^n \inner{x}{v_j}^2\lambda_j \geq 0
\end{equation*}
and thus $A$ is positive definite.

\section*{Problem 8.3}
Suppose that $u \in H^2(\Omega)$, for a bounded lipschitz boundary $\Omega \subseteq \R^d$. Then we have the Sobolev Embedding Theorem that for $j \geq 0$ and $m \geq 1$ that
\begin{equation*}
  W^{m + j, 2}(\Omega) \hookrightarrow W^{j,q}(\Omega)
\end{equation*}
for finitely many $q \leq 2d/(d-2m)$. Note that we can rewrite the bound as
\begin{equation*}
\frac{2d}{d-2m} = \frac{2d}{d - 2(2-j)} = \frac{2d}{d - 4 + 2j} 
\end{equation*}
then in order to have the largest possible bound for $q$, we choose $j = 0$ and have
\begin{equation*}
  q \leq \frac{2d}{d - 4}
\end{equation*}
and thus we have that
\begin{equation*}
  q^* = \begin{cases}
    \infty & d < 4 \\
    \frac{2d}{d - 4} & d \geq 4
  \end{cases}
\end{equation*}
Now note that by Holder's inequality we have that
\begin{equation*}
  \norm{cu^2}_{L^2} \leq \norm{c}_{L^p} \norm{u^2}_{L^q'}
\end{equation*}
for $1/p + 1/q' = 1/2$. Note that $\norm{u}_{L^q} < \infty$ for $q' = q/2$ and hence we have
\begin{equation*}
  \frac{1}{p} + \frac{1}{q'} = \frac{1}{p} + \frac{2}{q} =  \frac{1}{p} + \frac{2(d-4)}{2d} = \frac{1}{2}
\end{equation*}
and then we have $p = 2d/(8-d)$. Thus we have that
\begin{equation*}
  p^* = \begin{cases}
    2 & d \leq 4 \\
    \frac{2d}{8-d} & 4 < d < 8 \\
    \infty & d \geq 8
  \end{cases}
\end{equation*}
are the smallest possible values for $p$ such that $c \in L^{p^*}(\Omega)$.

\section*{Problem 8.4}
\subsection*{Part a}
Let $\gamma: H^1(\Omega) \to L^2(\Omega)$ be the trace operator, defined as $\gamma(u) = u|_V$ (which is continuous since $V$ has positive measure), then we see that $\text{ker}(\gamma) = H(\Omega)$ which is a closed subspace of $H^1(\Omega)$. Then equipping the space with the norm and inner production of $H^1(\Omega)$, we retrive that $H$ is a Hilbert space.

\subsection*{Part b}
To show that the Poincare inequality, we will assume that contrary, that is, suppose for $n \in \N$ there exists a function $u_n \in H^1(\Omega)$ such that
\begin{equation*}
  \norm{u_n}_{L^2(\Omega)} > n \norm{\nabla u_n}_{L^2(\Omega)}
\end{equation*}
and $\norm{u}_{L^2(\Omega)} = 1$. Then we have that
\begin{equation*}
  \norm{\nabla u_n}_{L^2(\Omega)} < \frac{1}{n}
\end{equation*}
Since $\norm{v_n}_{L^2(\Omega)} = 1$ and $\norm{\nabla u_n}_{L^2(\Omega)} < \frac{1}{n}$, we have that $\{u_n\}_{n=1}^\infty$ is a bounded sequence in $H^1(\Omega)$ and thus by the corollary of Rellich-Kondrachov we have that there exists a subsequence of $\{u_{n_k}\}_{k=1}^\infty$ such that $u_{n_k} \to u \in L^2(\Omega)$ in $L^2(\Omega)$. Since $\norm{\nabla u_{n_k}} \to 0$ as $n \to \infty$ we have that $\nabla u = 0$, and hence we have that $v \equiv C$ for some constant $C$ on $\Omega$. Then since each $u_n \in H$ we have that $u_n|_V = 0$ and hence we have that $u|_V = 0$, but since we found that $u$ is a constant on $\Omega$, we get that $u \equiv 0$ on all of $\Omega$. However,
\begin{equation*}
  \norm{u_n}_{L^2(\Omega)} = 1 \quad \text{but} \quad \norm{u_n}_{L^2(\Omega)} \to \norm{u}_{L^2(\Omega)} = 0
\end{equation*}
which is a contradiction and thus the Poincare inequality holds.
\end{document}
