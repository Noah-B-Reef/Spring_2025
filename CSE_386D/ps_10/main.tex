\documentclass[12pt]{report}
\usepackage{scribe,graphicx,graphics}
\usepackage{bbm}
\usepackage{yhmath}
\usepackage{stackengine}
\newcommand\dhookrightarrow{\mathrel{%
  \ensurestackMath{\stackanchor[.1ex]{\hookrightarrow}{\hookrightarrow}}
}}
\newcommand{\norm}[1]{\left|\left|#1\right|\right|}
\newcommand{\inner}[2]{\left\langle#1,#2\right\rangle}

\course{CSE 386D} 	
\coursetitle{Methods of Applied Mathematics II}	
\semester{Spring 2025}
\lecturer{} % Due Date: {\bf Mon, Oct 3 2016}}
\lecturetitle{Problem Set}
\lecturenumber{10}   
\lecturedate{}    
\input{commands.tex}

% Insert yousupp}
\scribe{Student Name: Noah Reef}

\begin{document}
\maketitle

\section*{Problem 8.14}
\subsection*{Part a}
Let $\mathcal{H} = H_0^1(\Omega)$ and $H = V_n$ for some $n$, and define
\begin{align*}
  B(u_n,v_n) &= \inner{\nabla u_n}{\nabla v_n}_{L^2(\Omega)} + \inner{u_n}{v_n}_{L^2(\Omega)}\\
  F(v_n) &= \inner{f}{v_n}_{L^2(\Omega)}
\end{align*}
Note that,
\begin{equation*}
  |F(v_n)| = \left|\inner{f}{v_n}_{L^2(\Omega)}\right| \leq \norm{f}_{L^2(\Omega)} \norm{v_n}_{L^2(\Omega)} \leq C_p \norm{f}_{H^1(\Omega)} \norm{v_n}_{H^1(\Omega)}
\end{equation*}
thus $F$ is continuous. Then we see that,
\begin{align*}
|B(u_n,v_n)| &= \left|\inner{\nabla u_n}{\nabla v_n}_{L^2(\Omega)} + \inner{u_n}{v_n}_{L^2(\Omega)}\right| \\
             &\leq \left|\inner{\nabla u_n}{\nabla v_n}_{L^2(\Omega)}\right| + \left|\inner{u_n}{v_n}_{L^2(\Omega)}\right| \\
             &\leq \norm{\nabla u_n}_{L^2(\Omega)} \norm{\nabla v_n}_{L^2(\Omega)} + \norm{u_n}_{L^2(\Omega)} \norm{v_n}_{L^2(\Omega)}\\
             &\leq (C_p + 1) \norm{u_n}_{H^1(\Omega)} \norm{v_n}_{H^1(\Omega)}\\
\end{align*}
thus $B$ is continous. Similarly, we have that 
\begin{align*}
  B(u_n,u_n) &= \inner{\nabla u_n}{\nabla u_n}_{L^2(\Omega)} + \inner{u_n}{u_n}_{L^2(\Omega)}\\
  &\geq \norm{\nabla u_n}_{L^2(\Omega)}^2 + \norm{u_n}_{L^2(\Omega)}^2\\
  &\geq  \norm{u_n}_{H^1(\Omega)}^2
\end{align*}
and hence $B$ si coercive. Thus by the Lax-Milgram theorem, we have that there exists a unique solution $u_n \in V_n$ such that
\begin{align*}
  B(u_n,v_n) &= F(v_n) \quad \forall v_n \in V_n
\end{align*}
and we see that
\begin{equation*}
|B(u_n,u_n)| = \norm{u_n}_{H^1(\Omega)}^2 = |\inner{f}{u_n}_{L^2(\Omega)}| \leq \norm{f}_{L^2(\Omega)} \norm{u_n}_{L^2(\Omega)} 
\end{equation*}
dividing by $\norm{u_n}_{L^2(\Omega)}$ gives us
\begin{equation*}
  \norm{u_n}_{H^1(\Omega)} \leq \norm{f}_{L^2(\Omega)} 
\end{equation*}

\subsection*{Part b}
From part a, we have that $u_n$ is uniformly bounded in $H^1(\Omega)$, thus by the Banach-Alaoglu theorem we have that there exists a subsequence $u_n \rightharpoonup u$ in $H^1(\Omega)$.
Note that $V = \bigcup_{n=1}^\infty V_n$ is dense in $H^1(\Omega)$, and hence $\bar{V} = H^1(\Omega)$. Since the variational problem, for each $n$, of finding $u_n \in V_n$ such that 
\begin{equation*}
  B(u_n, v_n) = F(v_n) \quad \forall v_n \in V_n
\end{equation*}
has a unique solution. We have that the same problem posed on $H^1(\Omega)$, also has a unique solution, that is
\begin{equation*}
  B(u,v) = F(v) \quad \forall v \in H^1(\Omega)
\end{equation*}
has a unique solution $u^* \in H^1(\Omega)$. For each $v \in H^1(\Omega)$, there exists a sequence $v_n \to v$ where $v_n \in V_n$, and hence we have that since $u_n \rightharpoonup u$ we get that
\begin{equation*}
B(u_n, v_n) = F(v_n) \to B(u,v) = F(v) \quad \forall v \in H^1(\Omega)
\end{equation*}
which implies that $u$ is a solution to the variational problem posed on $H^1(\Omega)$, and hence $u = u^*$.

\subsection*{Part c}
Note that 
\begin{equation*}
  \norm{u - u_n}_{H^1(\Omega)} \leq \frac{M}{\gamma} \inf_{v_n \in V_n} \norm{u - v_n}_{H^1(\Omega)}
\end{equation*}
where $M$ and $\gamma$ are the continuity and coercivity constants of $B$ respectively. Note that since $V_1 \subseteq V_2 \subseteq \dots$, we get that
\begin{equation*}
  \inf_{v_n \in V_n} \norm{u - v_n}_{H^1(\Omega)} \geq \inf_{v_n \in V_{n+1}} \norm{u-v_n}_{H^1(\Omega)}
\end{equation*}
for all $n$. Additionally since $V_n \to V$ as $n \to \infty$, which as stated above is dense in $H^1(\Omega)$, we get that
\begin{equation*}
  \inf_{v_n \in V_n} \norm{u - v_n}_{H^1(\Omega)} \to 0 \quad \text{as } n \to \infty
\end{equation*}
Thus we have that
\begin{equation*}
  \norm{u - u_n}_{H^1(\Omega)} \leq \frac{M}{\gamma} \inf_{v_n \in V_n} \norm{u - v_n}_{H^1(\Omega)} \to 0
\end{equation*}
monotically as $n \to \infty$.

\subsection*{Part d}
Recall that the variational formulation of the problem is given by
\begin{equation*}
  \inner{\nabla u_n}{\nabla v_n}_{L^2(\Omega)} + \inner{u_n}{v_n}_{L^2(\Omega)} = \inner{f}{v_n}_{L^2(\Omega)} \quad \forall v_n \in V_n
\end{equation*}
and then for the $\inner{\nabla u_n}{\nabla v_n}_{L^2(\Omega)}$ term, we have that 
\begin{equation*}
  \int_{\Omega} \nabla u \cdot \nabla v \, dx = - \int_{\Omega} (\nabla^2 u)v \, \dx + \int_{\partial \Omega}(\nabla u \cdot \nu)v \, d\sigma(x)
\end{equation*}
which implies that $u \in H^2(\Omega)$ and we get that
\begin{equation*}
  - \int_{\Omega} (\nabla^2 u)v \, \dx + \int_{\partial \Omega}(\nabla u \cdot \nu)v \, d\sigma(x) + \int_{\Omega} uv \, dx = \int_{\Omega} fv \, dx 
\end{equation*}
which holds for all $v \in H^1(\Omega)$, and hence we have that $\nabla u \cdot \nu = 0$ on $\partial \Omega$.

\section*{Problem 8.17}
Notice that $I_h$ is well-defined since for any $\{v(x_j)\}_{j=1}^{n-1}$ there exists only one line that passes through the points $(x_k, v(x_k))$ and $(x_{k+1},v(x_{k+1}))$jjk'. We can see that $I_h$ is linear since
Since $\Omega = (0,1)$ is bounded then we have by the Sobolev Embedding theorem that $H_0^1(0,1) \hookrightarrow C_B^0(0,1)$, that means that there exists $C > 0$ such that for all $u \in H_0^1(0,1)$
\begin{equation*}
  \norm{u}_{C_B^0(0,1)} \leq C \norm{u}_{H_0^1(0,1)}
\end{equation*}
so then for $v \in H_0^1(0,1)$ we have that 
\begin{equation*}
  \norm{\mathcal{I}_h v}_{C_B^0(0,1)} = \norm{v}\leq C \norm{\mathcal{I}_h v}_{H_0^1(0,1)} \leq C \norm{v}_{H_0^1(0,1)}
\end{equation*}
and hence we have that $\mathcal{I}_h$ is continuous.
\end{document}