\documentclass[12pt]{report}
\usepackage{scribe,graphicx,graphics}
\usepackage{bbm}
\usepackage{yhmath}
\usepackage{stackengine}
\newcommand\dhookrightarrow{\mathrel{%
  \ensurestackMath{\stackanchor[.1ex]{\hookrightarrow}{\hookrightarrow}}
}}
\newcommand{\norm}[1]{\left|\left|#1\right|\right|}
\newcommand{\inner}[2]{\left\langle#1,#2\right\rangle}

\course{CSE 386D} 	
\coursetitle{Methods of Applied Mathematics II}	
\semester{Spring 2025}
\lecturer{} % Due Date: {\bf Mon, Oct 3 2016}}
\lecturetitle{Problem Set}
\lecturenumber{7}   
\lecturedate{}    
\usepackage{enumerate}
\newcommand{\remind}[1]{\textcolor{red}{\textbf{#1}}} %To remind me of unfinished work to fix later
\newcommand{\hide}[1]{} %To hide large blocks of code without using % symbols

\newcommand{\ep}{\varepsilon}
\newcommand{\vp}{\varphi}
\newcommand{\lam}{\lambda}
\newcommand{\Lam}{\Lambda}
%\newcommand{\abs}[1]{\ensuremath{\left\lvert#1\right\rvert}} % This clashes with the physics package
%\newcommand{\norm}[1]{\ensuremath{\left\lVert#1\right\rVert}} % This clashes with the physics package
\newcommand{\floor}[1]{\ensuremath{\left\lfloor#1\right\rfloor}}
\newcommand{\ceil}[1]{\ensuremath{\left\lceil#1\right\rceil}}
\newcommand{\A}{\mathbb{A}}
\newcommand{\B}{\mathbb{B}}
\newcommand{\C}{\mathbb{C}}
\newcommand{\D}{\mathbb{D}}
\newcommand{\E}{\mathbb{E}}
\newcommand{\F}{\mathbb{F}}
\newcommand{\K}{\mathbb{K}}
\newcommand{\N}{\mathbb{N}}
\newcommand{\Q}{\mathbb{Q}}
\newcommand{\R}{\mathbb{R}}
\newcommand{\T}{\mathbb{T}}
\newcommand{\X}{\mathbb{X}}
\newcommand{\Y}{\mathbb{Y}}
\newcommand{\Z}{\mathbb{Z}}
\newcommand{\As}{\mathcal{A}}
\newcommand{\Bs}{\mathcal{B}}
\newcommand{\Cs}{\mathcal{C}}
\newcommand{\Ds}{\mathcal{D}}
\newcommand{\Es}{\mathcal{E}}
\newcommand{\Fs}{\mathcal{F}}
\newcommand{\Gs}{\mathcal{G}}
\newcommand{\Hs}{\mathcal{H}}
\newcommand{\Is}{\mathcal{I}}
\newcommand{\Js}{\mathcal{J}}
\newcommand{\Ks}{\mathcal{K}}
\newcommand{\Ls}{\mathcal{L}}
\newcommand{\Ms}{\mathcal{M}}
\newcommand{\Ns}{\mathcal{N}}
\newcommand{\Os}{\mathcal{O}}
\newcommand{\Ps}{\mathcal{P}}
\newcommand{\Qs}{\mathcal{Q}}
\newcommand{\Rs}{\mathcal{R}}
\newcommand{\Ss}{\mathcal{S}}
\newcommand{\Ts}{\mathcal{T}}
\newcommand{\Us}{\mathcal{U}}
\newcommand{\Vs}{\mathcal{V}}
\newcommand{\Ws}{\mathcal{W}}
\newcommand{\Xs}{\mathcal{X}}
\newcommand{\Ys}{\mathcal{Y}}
\newcommand{\Zs}{\mathcal{Z}}
\newcommand{\ab}{\textbf{a}}
\newcommand{\bb}{\textbf{b}}
\newcommand{\cb}{\textbf{c}}
\newcommand{\db}{\textbf{d}}
\newcommand{\ub}{\textbf{u}}
\newcommand{\sbb}{\textbf{s}}
%\renewcommand{\vb}{\textbf{v}} % This clashes with the physics package (the physics package already defines the \vb command)
\newcommand{\wb}{\textbf{w}}
\newcommand{\xb}{\textbf{x}}
\newcommand{\yb}{\textbf{y}}
\newcommand{\zb}{\textbf{z}}
\newcommand{\vbb}{\textbf{v}}
\newcommand{\Ab}{\textbf{A}}
\newcommand{\Bb}{\textbf{B}}
\newcommand{\Cb}{\textbf{C}}
\newcommand{\Db}{\textbf{D}}
\newcommand{\eb}{\textbf{e}}
\newcommand{\ex}{\textbf{e}_x}
\newcommand{\ey}{\textbf{e}_y}
\newcommand{\ez}{\textbf{e}_z}
\newcommand{\zerob}{\mathbf{0}}
\newcommand{\abar}{\overline{a}}
\newcommand{\bbar}{\overline{b}}
\newcommand{\cbar}{\overline{c}}
\newcommand{\dbar}{\overline{d}}
\newcommand{\ubar}{\overline{u}}
\newcommand{\vbar}{\overline{v}}
\newcommand{\wbar}{\overline{w}}
\newcommand{\xbar}{\overline{x}}
\newcommand{\ybar}{\overline{y}}
\newcommand{\zbar}{\overline{z}}
\newcommand{\Abar}{\overline{A}}
\newcommand{\Bbar}{\overline{B}}
\newcommand{\Cbar}{\overline{C}}
\newcommand{\Dbar}{\overline{D}}
\newcommand{\Ubar}{\overline{U}}
\newcommand{\Vbar}{\overline{V}}
\newcommand{\Wbar}{\overline{W}}
\newcommand{\Xbar}{\overline{X}}
\newcommand{\Ybar}{\overline{Y}}
\newcommand{\Zbar}{\overline{Z}}
\newcommand{\Aint}{A^\circ}
\newcommand{\Bint}{B^\circ}
\newcommand{\limk}{\lim_{k\to\infty}}
\newcommand{\limm}{\lim_{m\to\infty}}
\newcommand{\limn}{\lim_{n\to\infty}}
\newcommand{\limx}[1][a]{\lim_{x\to#1}}
\newcommand{\liminfm}{\liminf_{m\to\infty}}
\newcommand{\limsupm}{\limsup_{m\to\infty}}
\newcommand{\liminfn}{\liminf_{n\to\infty}}
\newcommand{\limsupn}{\limsup_{n\to\infty}}
\newcommand{\sumkn}{\sum_{k=1}^n}
\newcommand{\sumk}[1][1]{\sum_{k=#1}^\infty}
\newcommand{\summ}[1][1]{\sum_{m=#1}^\infty}
\newcommand{\sumn}[1][1]{\sum_{n=#1}^\infty}
\newcommand{\emp}{\varnothing}
\newcommand{\exc}{\backslash}
\newcommand{\sub}{\subseteq}
\newcommand{\sups}{\supseteq}
\newcommand{\capp}{\bigcap}
\newcommand{\cupp}{\bigcup}
\newcommand{\kupp}{\bigsqcup}
\newcommand{\cappkn}{\bigcap_{k=1}^n}
\newcommand{\cuppkn}{\bigcup_{k=1}^n}
\newcommand{\kuppkn}{\bigsqcup_{k=1}^n}
\newcommand{\cappk}[1][1]{\bigcap_{k=#1}^\infty}
\newcommand{\cuppk}[1][1]{\bigcup_{k=#1}^\infty}
\newcommand{\cappm}[1][1]{\bigcap_{m=#1}^\infty}
\newcommand{\cuppm}[1][1]{\bigcup_{m=#1}^\infty}
\newcommand{\cappn}[1][1]{\bigcap_{n=#1}^\infty}
\newcommand{\cuppn}[1][1]{\bigcup_{n=#1}^\infty}
\newcommand{\kuppk}[1][1]{\bigsqcup_{k=#1}^\infty}
\newcommand{\kuppm}[1][1]{\bigsqcup_{m=#1}^\infty}
\newcommand{\kuppn}[1][1]{\bigsqcup_{n=#1}^\infty}
\newcommand{\cappa}{\bigcap_{\alpha\in I}}
\newcommand{\cuppa}{\bigcup_{\alpha\in I}}
\newcommand{\kuppa}{\bigsqcup_{\alpha\in I}}
\newcommand{\Rx}{\overline{\mathbb{R}}}
\newcommand{\dx}{\,dx}
\newcommand{\dy}{\,dy}
\newcommand{\dt}{\,dt}
\newcommand{\dax}{\,d\alpha(x)}
\newcommand{\dbx}{\,d\beta(x)}
\DeclareMathOperator{\glb}{\text{glb}}
\DeclareMathOperator{\lub}{\text{lub}}
\newcommand{\xh}{\widehat{x}}
\newcommand{\yh}{\widehat{y}}
\newcommand{\zh}{\widehat{z}}
\newcommand{\<}{\langle}
\renewcommand{\>}{\rangle}
\renewcommand{\iff}{\Leftrightarrow}
\DeclareMathOperator{\im}{\text{im}}
\let\spn\relax\let\Re\relax\let\Im\relax
\DeclareMathOperator{\spn}{\text{span}}
\DeclareMathOperator{\sym}{\text{Sym}}
\DeclareMathOperator{\myskew}{\text{Skew}}
\DeclareMathOperator{\Re}{\text{Re}}
\DeclareMathOperator{\Im}{\text{Im}}
\DeclareMathOperator{\diag}{\text{diag}}
\endinput
\newcommand{\supp}[]{\text{supp}}

% Insert yousupp}
\scribe{Student Name: Noah Reef}

\begin{document}
\maketitle

\section*{Problem 7.12}
\subsection*{Part a}
Consider the integral,
\begin{align*}
  \int_{\R^d}(1 + |\xi|^2)^{-s} \, d\xi &= \int_{S_1(0)}\int_0^\infty (1 + r^2)^{-s}r^{d-1} \, dr \, d\omega. \\
                                        &\leq \omega_d \int_0^\infty r^(-2s) r^{d-1} \, dr \\ 
                                        &= \omega_d \int_0^\infty r^{d-2s-1} \, dr
\end{align*}
and since $s > d/2$, we have that $d - 2s -1 < d-d - 1 < - 1$. Thus, the integral converges. 

\subsection*{Part b}
Let $\phi \in \mathcal{S}$, then we see for $x \in \R^d$ that
\begin{align*}
  \phi(x) = (2\pi)^{-d/2} \int_{\R^d} e^{ix \cdot \xi} \hat{\phi}(\xi) \, d\xi = (2\pi)^{-d/2} \int_{\R^d} e^{ix \cdot \xi} (1 + |\xi|^2)^{-s/2}(1+ |\xi|^2)^{s/2} \hat{\phi}(\xi) \, d\xi.
\end{align*}
and get that
\begin{align*}
  |\phi(x)|^2 \leq (2\pi)^{-d} \int_{\R^d} (1+|\xi|^2)^{s} |\hat{\phi}(\xi)|^2 \, d\xi \int_{\R^d} (1 + |\xi|^2)^{-s} \, d\xi = C \norm{u}_{H^s}^2.
\end{align*}
where 
\begin{equation*}
  C = (2\pi)^{-d} \int_{\R^d} (1+|\xi|^2)^{-s} \, d\xi.
\end{equation*}

\subsection*{Part c}
Note that $\mathcal{S}$ is dense in $H^s(\R^d)$, and hence for any $\phi \in H^s(\R^d)$, there exists a sequence of $\phi_n \in \mathcal{S}$ such that $\phi_n \to \phi$ in $H^s(\R^d)$. Additionally for each $\phi_n \in \mathcal{S}$ we have the result above that
\begin{equation*}
  \norm{\phi_n}_{L^\infty(\R^d)} \leq C \norm{\phi_n}_{H^s(\R^d)}.
\end{equation*}
and hence we have that
\begin{equation*}
  \phi_{L^\infty(\R^d)} \leq C \norm{\phi}_{H^s(\R^d)}.
\end{equation*}
for $\phi \in H^s(\R^d)$. This proves the goal result that $H^s(\R^d) \hookrightarrow C_B^0(\R^d)$.

\section*{Problem 7.13}
\subsection*{Part a}
Let $f \in H^1(\R^d)$, then we see for $0 \leq s \leq 1$ we get that
\begin{align*}
  \norm{f}^2_{H^s(\R^d)} = \int_{\R^d} (1 + |\xi|^2)^2 |\hat{f}(\xi)|^2 \, d\xi &= \int_{\R^d} \left((1 + |\xi|^2) |\hat{f}(\xi)|^2\right)^s \left(|\hat{f}(\xi)|^2\right)^{1-s} \, d\xi \\
                                                                                &\leq \left(\int_{\R^d} (1 + |\xi|^2) |\hat{f}(\xi)|^2 \, d\xi\right)^s \left(\int_{\R^d} |\hat{f}(\xi)|^2 \, d\xi\right)^{1-s} \\
                                                                                &= \norm{f}^{2s}_{H^1(\R^d)} \norm{f}^{2(1-s)}_{L^2(\R^d)}
\end{align*}
In general, we have that if $f \in H^r(\R^d)$ and $0 \leq s \leq 1$, we get that 
\begin{align*}
  \norm{f}^2_{H^{rs}(\R^d)} = \int_{\R^d} (1 + |\xi|^2)^{rs} |\hat{f}(\xi)|^2 \, d\xi &= \int_{\R^d} \left((1 + |\xi|^2)^r |\hat{f}(\xi)|^2\right)^{s} \left(|\hat{f}(\xi)|^2\right)^{1-s} \, d\xi \\
                                                                                      &\leq \left(\int_{\R^d} (1 + |\xi|^2)^r |\hat{f}(\xi)|^2 \, d\xi\right)^s \left(\int_{\R^d} |\hat{f}(\xi)|^2 \, d\xi\right)^{1-s} \\
                                                                                      &= \norm{f}^{2s}_{H^r(\R^d)} \norm{f}^{2(1-s)}_{L^2(\R^d)}
\end{align*}
\subsection*{Part b}
Let $\Omega \subset \R^d$ be a bounded domain with smooth boundary $\partial \Omega$. We can show this by using the trace theorem and showing that
\begin{equation*}
  \norm{f}_{L^2(\partial \Omega)} = \norm{\gamma_0 f}_{H^{0}(\partial \Omega)} \leq C \norm{f}_{H^{1/2}(\Omega)} \leq \norm{f}^{1/2}_{H^1(\Omega)} \norm{f}^{1/2}_{L^2(\Omega)}
\end{equation*}
\end{document}
