\documentclass[12pt]{report}
\usepackage{scribe,graphicx,graphics}
\usepackage{bbm}
\usepackage{yhmath}
\usepackage{stackengine}
\newcommand\dhookrightarrow{\mathrel{%
  \ensurestackMath{\stackanchor[.1ex]{\hookrightarrow}{\hookrightarrow}}
}}
\newcommand{\norm}[1]{\left|\left|#1\right|\right|}
\newcommand{\inner}[2]{\left\langle#1,#2\right\rangle}

\course{CSE 386D} 	
\coursetitle{Methods of Applied Mathematics II}	
\semester{Spring 2025}
\lecturer{} % Due Date: {\bf Mon, Oct 3 2016}}
\lecturetitle{Problem Set}
\lecturenumber{7}   
\lecturedate{}    
\input{commands.tex}
\newcommand{\supp}[]{\text{supp}}

% Insert yousupp}
\scribe{Student Name: Noah Reef}

\begin{document}
\maketitle

\section*{Problem 7.12}
\subsection*{Part a}
Consider the integral,
\begin{align*}
  \int_{\R^d}(1 + |\xi|^2)^{-s} \, d\xi &= \int_{S_1(0)}\int_0^\infty (1 + r^2)^{-s}r^{d-1} \, dr \, d\omega. \\
                                        &\leq \omega_d \int_0^\infty r^(-2s) r^{d-1} \, dr \\ 
                                        &= \omega_d \int_0^\infty r^{d-2s-1} \, dr
\end{align*}
and since $s > d/2$, we have that $d - 2s -1 < d-d - 1 < - 1$. Thus, the integral converges. 

\subsection*{Part b}
Let $\phi \in \mathcal{S}$, then we see for $x \in \R^d$ that
\begin{align*}
  \phi(x) = (2\pi)^{-d/2} \int_{\R^d} e^{ix \cdot \xi} \hat{\phi}(\xi) \, d\xi = (2\pi)^{-d/2} \int_{\R^d} e^{ix \cdot \xi} (1 + |\xi|^2)^{-s/2}(1+ |\xi|^2)^{s/2} \hat{\phi}(\xi) \, d\xi.
\end{align*}
and get that
\begin{align*}
  |\phi(x)|^2 \leq (2\pi)^{-d} \int_{\R^d} (1+|\xi|^2)^{s} |\hat{\phi}(\xi)|^2 \, d\xi \int_{\R^d} (1 + |\xi|^2)^{-s} \, d\xi = C \norm{u}_{H^s}^2.
\end{align*}
where 
\begin{equation*}
  C = (2\pi)^{-d} \int_{\R^d} (1+|\xi|^2)^{-s} \, d\xi.
\end{equation*}

\subsection*{Part c}
Note that $\mathcal{S}$ is dense in $H^s(\R^d)$, and hence for any $\phi \in H^s(\R^d)$, there exists a sequence of $\phi_n \in \mathcal{S}$ such that $\phi_n \to \phi$ in $H^s(\R^d)$. Additionally for each $\phi_n \in \mathcal{S}$ we have the result above that
\begin{equation*}
  \norm{\phi_n}_{L^\infty(\R^d)} \leq C \norm{\phi_n}_{H^s(\R^d)}.
\end{equation*}
and hence we have that
\begin{equation*}
  \phi_{L^\infty(\R^d)} \leq C \norm{\phi}_{H^s(\R^d)}.
\end{equation*}
for $\phi \in H^s(\R^d)$. This proves the goal result that $H^s(\R^d) \hookrightarrow C_B^0(\R^d)$.

\section*{Problem 7.13}
\subsection*{Part a}
Let $f \in H^1(\R^d)$, then we see for $0 \leq s \leq 1$ we get that
\begin{align*}
  \norm{f}^2_{H^s(\R^d)} = \int_{\R^d} (1 + |\xi|^2)^2 |\hat{f}(\xi)|^2 \, d\xi &= \int_{\R^d} \left((1 + |\xi|^2) |\hat{f}(\xi)|^2\right)^s \left(|\hat{f}(\xi)|^2\right)^{1-s} \, d\xi \\
                                                                                &\leq \left(\int_{\R^d} (1 + |\xi|^2) |\hat{f}(\xi)|^2 \, d\xi\right)^s \left(\int_{\R^d} |\hat{f}(\xi)|^2 \, d\xi\right)^{1-s} \\
                                                                                &= \norm{f}^{2s}_{H^1(\R^d)} \norm{f}^{2(1-s)}_{L^2(\R^d)}
\end{align*}
In general, we have that if $f \in H^r(\R^d)$ and $0 \leq s \leq 1$, we get that 
\begin{align*}
  \norm{f}^2_{H^{rs}(\R^d)} = \int_{\R^d} (1 + |\xi|^2)^{rs} |\hat{f}(\xi)|^2 \, d\xi &= \int_{\R^d} \left((1 + |\xi|^2)^r |\hat{f}(\xi)|^2\right)^{s} \left(|\hat{f}(\xi)|^2\right)^{1-s} \, d\xi \\
                                                                                      &\leq \left(\int_{\R^d} (1 + |\xi|^2)^r |\hat{f}(\xi)|^2 \, d\xi\right)^s \left(\int_{\R^d} |\hat{f}(\xi)|^2 \, d\xi\right)^{1-s} \\
                                                                                      &= \norm{f}^{2s}_{H^r(\R^d)} \norm{f}^{2(1-s)}_{L^2(\R^d)}
\end{align*}
\subsection*{Part b}
Let $\Omega \subset \R^d$ be a bounded domain with smooth boundary $\partial \Omega$. We can show this by using the trace theorem and showing that
\begin{equation*}
  \norm{f}_{L^2(\partial \Omega)} = \norm{\gamma_0 f}_{H^{0}(\partial \Omega)} \leq C \norm{f}_{H^{1/2}(\Omega)} \leq \norm{f}^{1/2}_{H^1(\Omega)} \norm{f}^{1/2}_{L^2(\Omega)}
\end{equation*}
\end{document}
