\documentclass[12pt]{report}
\usepackage{scribe,graphicx,graphics}

\newcommand{\FT}{\mathcal{F}}
\newcommand{\norm}[1]{\left\lVert#1\right\rVert}

\course{CSE 386D} 	
\coursetitle{MoAM II}	
\semester{Spring 2025}
\lecturer{} % Due Date: {\bf Mon, Oct 3 2016}}
\lecturetitle{Problem Set}
\lecturenumber{1}   
\lecturedate{}    
\usepackage{enumerate}
\newcommand{\remind}[1]{\textcolor{red}{\textbf{#1}}} %To remind me of unfinished work to fix later
\newcommand{\hide}[1]{} %To hide large blocks of code without using % symbols

\newcommand{\ep}{\varepsilon}
\newcommand{\vp}{\varphi}
\newcommand{\lam}{\lambda}
\newcommand{\Lam}{\Lambda}
%\newcommand{\abs}[1]{\ensuremath{\left\lvert#1\right\rvert}} % This clashes with the physics package
%\newcommand{\norm}[1]{\ensuremath{\left\lVert#1\right\rVert}} % This clashes with the physics package
\newcommand{\floor}[1]{\ensuremath{\left\lfloor#1\right\rfloor}}
\newcommand{\ceil}[1]{\ensuremath{\left\lceil#1\right\rceil}}
\newcommand{\A}{\mathbb{A}}
\newcommand{\B}{\mathbb{B}}
\newcommand{\C}{\mathbb{C}}
\newcommand{\D}{\mathbb{D}}
\newcommand{\E}{\mathbb{E}}
\newcommand{\F}{\mathbb{F}}
\newcommand{\K}{\mathbb{K}}
\newcommand{\N}{\mathbb{N}}
\newcommand{\Q}{\mathbb{Q}}
\newcommand{\R}{\mathbb{R}}
\newcommand{\T}{\mathbb{T}}
\newcommand{\X}{\mathbb{X}}
\newcommand{\Y}{\mathbb{Y}}
\newcommand{\Z}{\mathbb{Z}}
\newcommand{\As}{\mathcal{A}}
\newcommand{\Bs}{\mathcal{B}}
\newcommand{\Cs}{\mathcal{C}}
\newcommand{\Ds}{\mathcal{D}}
\newcommand{\Es}{\mathcal{E}}
\newcommand{\Fs}{\mathcal{F}}
\newcommand{\Gs}{\mathcal{G}}
\newcommand{\Hs}{\mathcal{H}}
\newcommand{\Is}{\mathcal{I}}
\newcommand{\Js}{\mathcal{J}}
\newcommand{\Ks}{\mathcal{K}}
\newcommand{\Ls}{\mathcal{L}}
\newcommand{\Ms}{\mathcal{M}}
\newcommand{\Ns}{\mathcal{N}}
\newcommand{\Os}{\mathcal{O}}
\newcommand{\Ps}{\mathcal{P}}
\newcommand{\Qs}{\mathcal{Q}}
\newcommand{\Rs}{\mathcal{R}}
\newcommand{\Ss}{\mathcal{S}}
\newcommand{\Ts}{\mathcal{T}}
\newcommand{\Us}{\mathcal{U}}
\newcommand{\Vs}{\mathcal{V}}
\newcommand{\Ws}{\mathcal{W}}
\newcommand{\Xs}{\mathcal{X}}
\newcommand{\Ys}{\mathcal{Y}}
\newcommand{\Zs}{\mathcal{Z}}
\newcommand{\ab}{\textbf{a}}
\newcommand{\bb}{\textbf{b}}
\newcommand{\cb}{\textbf{c}}
\newcommand{\db}{\textbf{d}}
\newcommand{\ub}{\textbf{u}}
\newcommand{\sbb}{\textbf{s}}
%\renewcommand{\vb}{\textbf{v}} % This clashes with the physics package (the physics package already defines the \vb command)
\newcommand{\wb}{\textbf{w}}
\newcommand{\xb}{\textbf{x}}
\newcommand{\yb}{\textbf{y}}
\newcommand{\zb}{\textbf{z}}
\newcommand{\vbb}{\textbf{v}}
\newcommand{\Ab}{\textbf{A}}
\newcommand{\Bb}{\textbf{B}}
\newcommand{\Cb}{\textbf{C}}
\newcommand{\Db}{\textbf{D}}
\newcommand{\eb}{\textbf{e}}
\newcommand{\ex}{\textbf{e}_x}
\newcommand{\ey}{\textbf{e}_y}
\newcommand{\ez}{\textbf{e}_z}
\newcommand{\zerob}{\mathbf{0}}
\newcommand{\abar}{\overline{a}}
\newcommand{\bbar}{\overline{b}}
\newcommand{\cbar}{\overline{c}}
\newcommand{\dbar}{\overline{d}}
\newcommand{\ubar}{\overline{u}}
\newcommand{\vbar}{\overline{v}}
\newcommand{\wbar}{\overline{w}}
\newcommand{\xbar}{\overline{x}}
\newcommand{\ybar}{\overline{y}}
\newcommand{\zbar}{\overline{z}}
\newcommand{\Abar}{\overline{A}}
\newcommand{\Bbar}{\overline{B}}
\newcommand{\Cbar}{\overline{C}}
\newcommand{\Dbar}{\overline{D}}
\newcommand{\Ubar}{\overline{U}}
\newcommand{\Vbar}{\overline{V}}
\newcommand{\Wbar}{\overline{W}}
\newcommand{\Xbar}{\overline{X}}
\newcommand{\Ybar}{\overline{Y}}
\newcommand{\Zbar}{\overline{Z}}
\newcommand{\Aint}{A^\circ}
\newcommand{\Bint}{B^\circ}
\newcommand{\limk}{\lim_{k\to\infty}}
\newcommand{\limm}{\lim_{m\to\infty}}
\newcommand{\limn}{\lim_{n\to\infty}}
\newcommand{\limx}[1][a]{\lim_{x\to#1}}
\newcommand{\liminfm}{\liminf_{m\to\infty}}
\newcommand{\limsupm}{\limsup_{m\to\infty}}
\newcommand{\liminfn}{\liminf_{n\to\infty}}
\newcommand{\limsupn}{\limsup_{n\to\infty}}
\newcommand{\sumkn}{\sum_{k=1}^n}
\newcommand{\sumk}[1][1]{\sum_{k=#1}^\infty}
\newcommand{\summ}[1][1]{\sum_{m=#1}^\infty}
\newcommand{\sumn}[1][1]{\sum_{n=#1}^\infty}
\newcommand{\emp}{\varnothing}
\newcommand{\exc}{\backslash}
\newcommand{\sub}{\subseteq}
\newcommand{\sups}{\supseteq}
\newcommand{\capp}{\bigcap}
\newcommand{\cupp}{\bigcup}
\newcommand{\kupp}{\bigsqcup}
\newcommand{\cappkn}{\bigcap_{k=1}^n}
\newcommand{\cuppkn}{\bigcup_{k=1}^n}
\newcommand{\kuppkn}{\bigsqcup_{k=1}^n}
\newcommand{\cappk}[1][1]{\bigcap_{k=#1}^\infty}
\newcommand{\cuppk}[1][1]{\bigcup_{k=#1}^\infty}
\newcommand{\cappm}[1][1]{\bigcap_{m=#1}^\infty}
\newcommand{\cuppm}[1][1]{\bigcup_{m=#1}^\infty}
\newcommand{\cappn}[1][1]{\bigcap_{n=#1}^\infty}
\newcommand{\cuppn}[1][1]{\bigcup_{n=#1}^\infty}
\newcommand{\kuppk}[1][1]{\bigsqcup_{k=#1}^\infty}
\newcommand{\kuppm}[1][1]{\bigsqcup_{m=#1}^\infty}
\newcommand{\kuppn}[1][1]{\bigsqcup_{n=#1}^\infty}
\newcommand{\cappa}{\bigcap_{\alpha\in I}}
\newcommand{\cuppa}{\bigcup_{\alpha\in I}}
\newcommand{\kuppa}{\bigsqcup_{\alpha\in I}}
\newcommand{\Rx}{\overline{\mathbb{R}}}
\newcommand{\dx}{\,dx}
\newcommand{\dy}{\,dy}
\newcommand{\dt}{\,dt}
\newcommand{\dax}{\,d\alpha(x)}
\newcommand{\dbx}{\,d\beta(x)}
\DeclareMathOperator{\glb}{\text{glb}}
\DeclareMathOperator{\lub}{\text{lub}}
\newcommand{\xh}{\widehat{x}}
\newcommand{\yh}{\widehat{y}}
\newcommand{\zh}{\widehat{z}}
\newcommand{\<}{\langle}
\renewcommand{\>}{\rangle}
\renewcommand{\iff}{\Leftrightarrow}
\DeclareMathOperator{\im}{\text{im}}
\let\spn\relax\let\Re\relax\let\Im\relax
\DeclareMathOperator{\spn}{\text{span}}
\DeclareMathOperator{\sym}{\text{Sym}}
\DeclareMathOperator{\myskew}{\text{Skew}}
\DeclareMathOperator{\Re}{\text{Re}}
\DeclareMathOperator{\Im}{\text{Im}}
\DeclareMathOperator{\diag}{\text{diag}}
\endinput

% Insert your name here!
\scribe{Student Name: Noah Reef}

\begin{document}
\maketitle

\section*{Problem 6.1}
To find the Fourier transform of $f(x) = e^{-|x|}$ for $x \in \mathbb{R}$, we compute
\begin{align*}
  \hat{f}(\xi) &= \frac{1}{\sqrt{2\pi}} \int_{-\infty}^\infty e^{-|x|}e^{-ix\xi} \, dx \\
               &= \frac{1}{\sqrt{2\pi}} \left(\int_{-\infty}^0 e^{x(1 - i\xi)} \, dx + \int_0^\infty e^{-x(1 + i\xi)} \, dx\right) \\
               &= \frac{1}{\sqrt{2\pi}} \left(\frac{1}{1-i\xi} e^{(x(1-i\xi))}\big\vert_{-\infty}^0 - \frac{1}{1 + i\xi} e^{-x(1+i\xi)}\big\vert_0^\infty\right) \\
               &= \frac{1}{\sqrt{2\pi}} \left( \frac{2}{(1-i\xi)(1+i\xi)}\right) \\
               &= \frac{1}{\sqrt{2\pi}} \left( \frac{2}{1+\xi^2}\right)
\end{align*}


\section*{Problem 6.2}
We will consider the parameteric curve defined as
\begin{align*}
  \gamma_1 &= t \quad &-R \leq t \leq R \\
  \gamma_2 &= R - it \quad &0 \leq t \leq \frac{\xi}{2a} \\
  \gamma_3 &= -t - i\frac{\xi}{2a} \quad &-R \leq t \leq R \\
  \gamma_4 &= -R + it \quad & -\frac{\xi}{2a} \leq t \leq 0
\end{align*}

and by Cauchy's Integral theorem we have that 
\begin{align*}
  0 &= \int_{-R}^R e^{-at^2 -i\xi t} \, dt + i\int_{0}^{\xi/2a} e^{-a(R - it)^2 - i\xi(R - it)} \, dt \\
                                   &- \int_{-R}^R e^{-a(-t - i\xi/2a)^2 - i(-t - i\xi/2a)} \, dt  - i\int_{-\xi/2a}^0 e^{-a(-R + it)^2 - i\xi(-R + it)} \, dt
\end{align*}
Then we see that the second and fourth terms go to zero since,
\begin{align*}
  \left\lvert i\int_{0}^{\xi/2a} e^{-a(R - it)^2 - i\xi(R - it)} \, dt \right\rvert + \left\lvert i\int_{-\xi/2a}^0 e^{-a(-R + it)^2 - i\xi(-R + it)} \, dt\right\rvert \leq 2\frac{\xi}{2a} e^{-aR^2} \to 0
\end{align*}
and we see for the third term that 
\begin{equation*}
  \lim_{R \to \infty} \int_{-R}^R e^{-a(-t - i\xi/2a)^2 - i(-t - i\xi/2a)} \, dt =\sqrt{\frac{\pi}{a}} e^{-\xi^2/(4a)}
\end{equation*}
and hence 
\begin{equation*}
  \hat{f}(\xi) = \frac{1}{\sqrt{2\pi}} \int_{\mathbb{R}} e^{-a|x|^2} e^{-ix\xi} \, dx = \sqrt{\frac{\pi}{a}} \frac{e^{-\xi^2/(4a)}}{\sqrt{2\pi}}
\end{equation*}
\section*{Problem 6.4}
Suppose the $f \in L^1(\mathbb{R}^d)$ and $f(x) = g(|x|)$ for some $g$, then we see that
\begin{align*}
  \hat{f}(\xi) &= (2\pi)^{-d/2} \int_{\mathbb{R}^d} f(x)e^{-i x \cdot \xi} \, dx \\
               &=(2\pi)^{-d/2} \int_{\mathbb{R}^d} g(|x|)e^{-i x \cdot \xi} \, dx \\
               &=(2\pi)^{-d/2} \int_{\mathbb{R}^d} g(|x|)e^{-i |x||\xi| \cos(\theta)} \, dx \\
               &=(2\pi)^{-d/2} \int_0^\infty \int_{\omega_d} g(r)e^{-i r|\xi| \cos(\theta)} r^{d} \, dr\, d\theta \\
               &= h(|\xi|)
\end{align*}

\section*{Problem 6.11}
Consider $\FT: L^1(\R^d) \ to C_v(\R^d)$. Recall that $\FT: L^1(\R^d) \to L^\infty(\R^d)$ is a bounded linear map by Proposition 6.2 and we know that $C_v(\R^d)$ is a closed linear subspace of $L^\infty(\R^d)$ by Proposition 6.4, hence $\FT: L^1(\R^d) \to C_v(\R^d)$ is a bounded linear map. Next we note that if $f,g \in L^1(\R^d)$ such that $\FT(f) = \FT(g)$ for all $\xi \in \R^d$, then we have that $\FT(f - g) = 0$ and hence $f - g = 0$ and so $f = g$, thus $\FT$ is injective. Now suppose that $\FT$ is surjective, then we have by the Open Mapping Theorem, that $\FT^{-1}$ is bounded. 

Next suppose we have the characteristic functions $f_n,f_1 \in L^1(\R^d)$ and consider $f_n * f_1 \in L^1(\R^d)$, then we see that
\begin{align*}
  f_n * f_1 &= \int_{R^d} f_n(x-y)f_1(y) \, dy \\
            &= \int_{[-1,1]^d}f_n(x-y) \, dy \\
            &= \int_{[x-1,x+1]^d} f_n(z) \. dz \\
            &= \int_{[x-1,0]^d} f_n(z) \, dz + \int_{[0,x+1]^d} f_n(z) \, dz \tag{$\in C_v(\R^d)$ By Exercise $5$} \\
            &= \begin{cases}
              x + n + 1 & x \in [-n - 1,-n + 1] \\
              2 & x \in [-n + 1,n - 1] \\
              n + 1 - x & x \in [n - 1,n + 1] \\
              0 & \text{otherwise}
            \end{cases} 
\end{align*}
then we see that as $n \to \infty$ we have that $f_n * f_1 \to 2$. Note that
\begin{equation*}
  \FT^{-1}(f_n * f_1) = (2\pi)^{-d/2}\frac{\sin(nx)\sin(x)}{x^2}
\end{equation*}
where $C$ is some constant. However as $n \to \infty$ we see that 
\begin{align*}
  \norm{\FT^{-1}[f_n * f_1]}_{L^1} &= (2\pi)^{-d/2}\int_{\R^d} \frac{|\sin(nx)\sin(x)|}{|x^2|} \, dx \\
                                   &\geq (2\pi)^{-d/2}\int_{[0,1]^d}\frac{|\sin(nx)\sin(x)|}{|x^2|} \, dx \\
                                   &\geq (2\pi)^{-d/2}\frac{2}{\pi} \int_{[0,1]^d} \left|\frac{\sin(nx)}{x}\right| \, dx  \to \infty
\end{align*}
which contradicts our assumption that $\FT^{-1}$ is bounded and hence $\FT$ is not surjective. Since $\FT: \mathcal{S} \to \mathcal{S}$ is a bounded linear map that is one-to-one and onto, such that $C_0^\infty(\R^d) \subsetneqq \mathcal{S} \subsetneqq L^1(\mathbb{R}^d)$ then we see that for $\phi \in C_0^\infty(\R^d)$ we have that $\FT^{-1}(\phi)$ exists and is in $L^1(\R^d)$. All there is left to show is that $C_0^\infty(\R^d)$ is dense in $C_v(\R^d)$. Consider $g \in C_v(\R^d)$,letting $\varphi_\epsilon$ be an approximation to the identity, defined as
\begin{equation*}
  \varphi_\epsilon(x) = \epsilon^{-d}\varphi(x/\epsilon)
\end{equation*}
with $\epsilon = 1/n$ and $\varphi \in C_0^\infty(\R^d)$, then we have that $g * \varphi_\epsilon \in C_0^\infty(\R^d)$ and $g * \varphi_\epsilon \to g$ in $C_v(\R^d)$ as $n \to \infty$ and hence $C_0^\infty(\R^d)$ is dense in $C_v(\R^d)$. 
\end{document}

