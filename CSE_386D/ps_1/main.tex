\documentclass[12pt]{report}
\usepackage{scribe,graphicx,graphics}

\newcommand{\FT}{\mathcal{F}}
\newcommand{\norm}[1]{\left\lVert#1\right\rVert}

\course{CSE 386D} 	
\coursetitle{MoAM II}	
\semester{Spring 2025}
\lecturer{} % Due Date: {\bf Mon, Oct 3 2016}}
\lecturetitle{Problem Set}
\lecturenumber{1}   
\lecturedate{}    
\input{commands.tex}

% Insert your name here!
\scribe{Student Name: Noah Reef}

\begin{document}
\maketitle

\section*{Problem 6.1}
To find the Fourier transform of $f(x) = e^{-|x|}$ for $x \in \mathbb{R}$, we compute
\begin{align*}
  \hat{f}(\xi) &= \frac{1}{\sqrt{2\pi}} \int_{-\infty}^\infty e^{-|x|}e^{-ix\xi} \, dx \\
               &= \frac{1}{\sqrt{2\pi}} \left(\int_{-\infty}^0 e^{x(1 - i\xi)} \, dx + \int_0^\infty e^{-x(1 + i\xi)} \, dx\right) \\
               &= \frac{1}{\sqrt{2\pi}} \left(\frac{1}{1-i\xi} e^{(x(1-i\xi))}\big\vert_{-\infty}^0 - \frac{1}{1 + i\xi} e^{-x(1+i\xi)}\big\vert_0^\infty\right) \\
               &= \frac{1}{\sqrt{2\pi}} \left( \frac{2}{(1-i\xi)(1+i\xi)}\right) \\
               &= \frac{1}{\sqrt{2\pi}} \left( \frac{2}{1+\xi^2}\right)
\end{align*}


\section*{Problem 6.2}
We will consider the parameteric curve defined as
\begin{align*}
  \gamma_1 &= t \quad &-R \leq t \leq R \\
  \gamma_2 &= R - it \quad &0 \leq t \leq \frac{\xi}{2a} \\
  \gamma_3 &= -t - i\frac{\xi}{2a} \quad &-R \leq t \leq R \\
  \gamma_4 &= -R + it \quad & -\frac{\xi}{2a} \leq t \leq 0
\end{align*}

and by Cauchy's Integral theorem we have that 
\begin{align*}
  0 &= \int_{-R}^R e^{-at^2 -i\xi t} \, dt + i\int_{0}^{\xi/2a} e^{-a(R - it)^2 - i\xi(R - it)} \, dt \\
                                   &- \int_{-R}^R e^{-a(-t - i\xi/2a)^2 - i(-t - i\xi/2a)} \, dt  - i\int_{-\xi/2a}^0 e^{-a(-R + it)^2 - i\xi(-R + it)} \, dt
\end{align*}
Then we see that the second and fourth terms go to zero since,
\begin{align*}
  \left\lvert i\int_{0}^{\xi/2a} e^{-a(R - it)^2 - i\xi(R - it)} \, dt \right\rvert + \left\lvert i\int_{-\xi/2a}^0 e^{-a(-R + it)^2 - i\xi(-R + it)} \, dt\right\rvert \leq 2\frac{\xi}{2a} e^{-aR^2} \to 0
\end{align*}
and we see for the third term that 
\begin{equation*}
  \lim_{R \to \infty} \int_{-R}^R e^{-a(-t - i\xi/2a)^2 - i(-t - i\xi/2a)} \, dt =\sqrt{\frac{\pi}{a}} e^{-\xi^2/(4a)}
\end{equation*}
and hence 
\begin{equation*}
  \hat{f}(\xi) = \frac{1}{\sqrt{2\pi}} \int_{\mathbb{R}} e^{-a|x|^2} e^{-ix\xi} \, dx = \sqrt{\frac{\pi}{a}} \frac{e^{-\xi^2/(4a)}}{\sqrt{2\pi}}
\end{equation*}
\section*{Problem 6.4}
Suppose the $f \in L^1(\mathbb{R}^d)$ and $f(x) = g(|x|)$ for some $g$, then we see that
\begin{align*}
  \hat{f}(\xi) &= (2\pi)^{-d/2} \int_{\mathbb{R}^d} f(x)e^{-i x \cdot \xi} \, dx \\
               &=(2\pi)^{-d/2} \int_{\mathbb{R}^d} g(|x|)e^{-i x \cdot \xi} \, dx \\
               &=(2\pi)^{-d/2} \int_{\mathbb{R}^d} g(|x|)e^{-i |x||\xi| \cos(\theta)} \, dx \\
               &=(2\pi)^{-d/2} \int_0^\infty \int_{\omega_d} g(r)e^{-i r|\xi| \cos(\theta)} r^{d} \, dr\, d\theta \\
               &= h(|\xi|)
\end{align*}

\section*{Problem 6.11}
Consider $\FT: L^1(\R^d) \ to C_v(\R^d)$. Recall that $\FT: L^1(\R^d) \to L^\infty(\R^d)$ is a bounded linear map by Proposition 6.2 and we know that $C_v(\R^d)$ is a closed linear subspace of $L^\infty(\R^d)$ by Proposition 6.4, hence $\FT: L^1(\R^d) \to C_v(\R^d)$ is a bounded linear map. Next we note that if $f,g \in L^1(\R^d)$ such that $\FT(f) = \FT(g)$ for all $\xi \in \R^d$, then we have that $\FT(f - g) = 0$ and hence $f - g = 0$ and so $f = g$, thus $\FT$ is injective. Now suppose that $\FT$ is surjective, then we have by the Open Mapping Theorem, that $\FT^{-1}$ is bounded. 

Next suppose we have the characteristic functions $f_n,f_1 \in L^1(\R^d)$ and consider $f_n * f_1 \in L^1(\R^d)$, then we see that
\begin{align*}
  f_n * f_1 &= \int_{R^d} f_n(x-y)f_1(y) \, dy \\
            &= \int_{[-1,1]^d}f_n(x-y) \, dy \\
            &= \int_{[x-1,x+1]^d} f_n(z) \. dz \\
            &= \int_{[x-1,0]^d} f_n(z) \, dz + \int_{[0,x+1]^d} f_n(z) \, dz \tag{$\in C_v(\R^d)$ By Exercise $5$} \\
            &= \begin{cases}
              x + n + 1 & x \in [-n - 1,-n + 1] \\
              2 & x \in [-n + 1,n - 1] \\
              n + 1 - x & x \in [n - 1,n + 1] \\
              0 & \text{otherwise}
            \end{cases} 
\end{align*}
then we see that as $n \to \infty$ we have that $f_n * f_1 \to 2$. Note that
\begin{equation*}
  \FT^{-1}(f_n * f_1) = (2\pi)^{-d/2}\frac{\sin(nx)\sin(x)}{x^2}
\end{equation*}
where $C$ is some constant. However as $n \to \infty$ we see that 
\begin{align*}
  \norm{\FT^{-1}[f_n * f_1]}_{L^1} &= (2\pi)^{-d/2}\int_{\R^d} \frac{|\sin(nx)\sin(x)|}{|x^2|} \, dx \\
                                   &\geq (2\pi)^{-d/2}\int_{[0,1]^d}\frac{|\sin(nx)\sin(x)|}{|x^2|} \, dx \\
                                   &\geq (2\pi)^{-d/2}\frac{2}{\pi} \int_{[0,1]^d} \left|\frac{\sin(nx)}{x}\right| \, dx  \to \infty
\end{align*}
which contradicts our assumption that $\FT^{-1}$ is bounded and hence $\FT$ is not surjective. Since $\FT: \mathcal{S} \to \mathcal{S}$ is a bounded linear map that is one-to-one and onto, such that $C_0^\infty(\R^d) \subsetneqq \mathcal{S} \subsetneqq L^1(\mathbb{R}^d)$ then we see that for $\phi \in C_0^\infty(\R^d)$ we have that $\FT^{-1}(\phi)$ exists and is in $L^1(\R^d)$. All there is left to show is that $C_0^\infty(\R^d)$ is dense in $C_v(\R^d)$. Consider $g \in C_v(\R^d)$,letting $\varphi_\epsilon$ be an approximation to the identity, defined as
\begin{equation*}
  \varphi_\epsilon(x) = \epsilon^{-d}\varphi(x/\epsilon)
\end{equation*}
with $\epsilon = 1/n$ and $\varphi \in C_0^\infty(\R^d)$, then we have that $g * \varphi_\epsilon \in C_0^\infty(\R^d)$ and $g * \varphi_\epsilon \to g$ in $C_v(\R^d)$ as $n \to \infty$ and hence $C_0^\infty(\R^d)$ is dense in $C_v(\R^d)$. 
\end{document}

