\documentclass[12pt]{report}
\usepackage{scribe,graphicx,graphics}
\usepackage{bbm}
\usepackage{yhmath}
\usepackage{stackengine}
\newcommand\dhookrightarrow{\mathrel{%
  \ensurestackMath{\stackanchor[.1ex]{\hookrightarrow}{\hookrightarrow}}
}}
\newcommand{\norm}[1]{\left|\left|#1\right|\right|}
\newcommand{\inner}[2]{\left\langle#1,#2\right\rangle}

\course{CSE 386D} 	
\coursetitle{Methods of Applied Mathematics II}	
\semester{Spring 2025}
\lecturer{} % Due Date: {\bf Mon, Oct 3 2016}}
\lecturetitle{Problem Set}
\lecturenumber{11}   
\lecturedate{}    
\usepackage{enumerate}
\newcommand{\remind}[1]{\textcolor{red}{\textbf{#1}}} %To remind me of unfinished work to fix later
\newcommand{\hide}[1]{} %To hide large blocks of code without using % symbols

\newcommand{\ep}{\varepsilon}
\newcommand{\vp}{\varphi}
\newcommand{\lam}{\lambda}
\newcommand{\Lam}{\Lambda}
%\newcommand{\abs}[1]{\ensuremath{\left\lvert#1\right\rvert}} % This clashes with the physics package
%\newcommand{\norm}[1]{\ensuremath{\left\lVert#1\right\rVert}} % This clashes with the physics package
\newcommand{\floor}[1]{\ensuremath{\left\lfloor#1\right\rfloor}}
\newcommand{\ceil}[1]{\ensuremath{\left\lceil#1\right\rceil}}
\newcommand{\A}{\mathbb{A}}
\newcommand{\B}{\mathbb{B}}
\newcommand{\C}{\mathbb{C}}
\newcommand{\D}{\mathbb{D}}
\newcommand{\E}{\mathbb{E}}
\newcommand{\F}{\mathbb{F}}
\newcommand{\K}{\mathbb{K}}
\newcommand{\N}{\mathbb{N}}
\newcommand{\Q}{\mathbb{Q}}
\newcommand{\R}{\mathbb{R}}
\newcommand{\T}{\mathbb{T}}
\newcommand{\X}{\mathbb{X}}
\newcommand{\Y}{\mathbb{Y}}
\newcommand{\Z}{\mathbb{Z}}
\newcommand{\As}{\mathcal{A}}
\newcommand{\Bs}{\mathcal{B}}
\newcommand{\Cs}{\mathcal{C}}
\newcommand{\Ds}{\mathcal{D}}
\newcommand{\Es}{\mathcal{E}}
\newcommand{\Fs}{\mathcal{F}}
\newcommand{\Gs}{\mathcal{G}}
\newcommand{\Hs}{\mathcal{H}}
\newcommand{\Is}{\mathcal{I}}
\newcommand{\Js}{\mathcal{J}}
\newcommand{\Ks}{\mathcal{K}}
\newcommand{\Ls}{\mathcal{L}}
\newcommand{\Ms}{\mathcal{M}}
\newcommand{\Ns}{\mathcal{N}}
\newcommand{\Os}{\mathcal{O}}
\newcommand{\Ps}{\mathcal{P}}
\newcommand{\Qs}{\mathcal{Q}}
\newcommand{\Rs}{\mathcal{R}}
\newcommand{\Ss}{\mathcal{S}}
\newcommand{\Ts}{\mathcal{T}}
\newcommand{\Us}{\mathcal{U}}
\newcommand{\Vs}{\mathcal{V}}
\newcommand{\Ws}{\mathcal{W}}
\newcommand{\Xs}{\mathcal{X}}
\newcommand{\Ys}{\mathcal{Y}}
\newcommand{\Zs}{\mathcal{Z}}
\newcommand{\ab}{\textbf{a}}
\newcommand{\bb}{\textbf{b}}
\newcommand{\cb}{\textbf{c}}
\newcommand{\db}{\textbf{d}}
\newcommand{\ub}{\textbf{u}}
\newcommand{\sbb}{\textbf{s}}
%\renewcommand{\vb}{\textbf{v}} % This clashes with the physics package (the physics package already defines the \vb command)
\newcommand{\wb}{\textbf{w}}
\newcommand{\xb}{\textbf{x}}
\newcommand{\yb}{\textbf{y}}
\newcommand{\zb}{\textbf{z}}
\newcommand{\vbb}{\textbf{v}}
\newcommand{\Ab}{\textbf{A}}
\newcommand{\Bb}{\textbf{B}}
\newcommand{\Cb}{\textbf{C}}
\newcommand{\Db}{\textbf{D}}
\newcommand{\eb}{\textbf{e}}
\newcommand{\ex}{\textbf{e}_x}
\newcommand{\ey}{\textbf{e}_y}
\newcommand{\ez}{\textbf{e}_z}
\newcommand{\zerob}{\mathbf{0}}
\newcommand{\abar}{\overline{a}}
\newcommand{\bbar}{\overline{b}}
\newcommand{\cbar}{\overline{c}}
\newcommand{\dbar}{\overline{d}}
\newcommand{\ubar}{\overline{u}}
\newcommand{\vbar}{\overline{v}}
\newcommand{\wbar}{\overline{w}}
\newcommand{\xbar}{\overline{x}}
\newcommand{\ybar}{\overline{y}}
\newcommand{\zbar}{\overline{z}}
\newcommand{\Abar}{\overline{A}}
\newcommand{\Bbar}{\overline{B}}
\newcommand{\Cbar}{\overline{C}}
\newcommand{\Dbar}{\overline{D}}
\newcommand{\Ubar}{\overline{U}}
\newcommand{\Vbar}{\overline{V}}
\newcommand{\Wbar}{\overline{W}}
\newcommand{\Xbar}{\overline{X}}
\newcommand{\Ybar}{\overline{Y}}
\newcommand{\Zbar}{\overline{Z}}
\newcommand{\Aint}{A^\circ}
\newcommand{\Bint}{B^\circ}
\newcommand{\limk}{\lim_{k\to\infty}}
\newcommand{\limm}{\lim_{m\to\infty}}
\newcommand{\limn}{\lim_{n\to\infty}}
\newcommand{\limx}[1][a]{\lim_{x\to#1}}
\newcommand{\liminfm}{\liminf_{m\to\infty}}
\newcommand{\limsupm}{\limsup_{m\to\infty}}
\newcommand{\liminfn}{\liminf_{n\to\infty}}
\newcommand{\limsupn}{\limsup_{n\to\infty}}
\newcommand{\sumkn}{\sum_{k=1}^n}
\newcommand{\sumk}[1][1]{\sum_{k=#1}^\infty}
\newcommand{\summ}[1][1]{\sum_{m=#1}^\infty}
\newcommand{\sumn}[1][1]{\sum_{n=#1}^\infty}
\newcommand{\emp}{\varnothing}
\newcommand{\exc}{\backslash}
\newcommand{\sub}{\subseteq}
\newcommand{\sups}{\supseteq}
\newcommand{\capp}{\bigcap}
\newcommand{\cupp}{\bigcup}
\newcommand{\kupp}{\bigsqcup}
\newcommand{\cappkn}{\bigcap_{k=1}^n}
\newcommand{\cuppkn}{\bigcup_{k=1}^n}
\newcommand{\kuppkn}{\bigsqcup_{k=1}^n}
\newcommand{\cappk}[1][1]{\bigcap_{k=#1}^\infty}
\newcommand{\cuppk}[1][1]{\bigcup_{k=#1}^\infty}
\newcommand{\cappm}[1][1]{\bigcap_{m=#1}^\infty}
\newcommand{\cuppm}[1][1]{\bigcup_{m=#1}^\infty}
\newcommand{\cappn}[1][1]{\bigcap_{n=#1}^\infty}
\newcommand{\cuppn}[1][1]{\bigcup_{n=#1}^\infty}
\newcommand{\kuppk}[1][1]{\bigsqcup_{k=#1}^\infty}
\newcommand{\kuppm}[1][1]{\bigsqcup_{m=#1}^\infty}
\newcommand{\kuppn}[1][1]{\bigsqcup_{n=#1}^\infty}
\newcommand{\cappa}{\bigcap_{\alpha\in I}}
\newcommand{\cuppa}{\bigcup_{\alpha\in I}}
\newcommand{\kuppa}{\bigsqcup_{\alpha\in I}}
\newcommand{\Rx}{\overline{\mathbb{R}}}
\newcommand{\dx}{\,dx}
\newcommand{\dy}{\,dy}
\newcommand{\dt}{\,dt}
\newcommand{\dax}{\,d\alpha(x)}
\newcommand{\dbx}{\,d\beta(x)}
\DeclareMathOperator{\glb}{\text{glb}}
\DeclareMathOperator{\lub}{\text{lub}}
\newcommand{\xh}{\widehat{x}}
\newcommand{\yh}{\widehat{y}}
\newcommand{\zh}{\widehat{z}}
\newcommand{\<}{\langle}
\renewcommand{\>}{\rangle}
\renewcommand{\iff}{\Leftrightarrow}
\DeclareMathOperator{\im}{\text{im}}
\let\spn\relax\let\Re\relax\let\Im\relax
\DeclareMathOperator{\spn}{\text{span}}
\DeclareMathOperator{\sym}{\text{Sym}}
\DeclareMathOperator{\myskew}{\text{Skew}}
\DeclareMathOperator{\Re}{\text{Re}}
\DeclareMathOperator{\Im}{\text{Im}}
\DeclareMathOperator{\diag}{\text{diag}}
\endinput

% Insert yousupp}
\scribe{Student Name: Noah Reef}

\begin{document}
\maketitle

\section*{Problem 9.8}
\subsection*{Part a}
Let $X = C^0([0,T])$ and define the operator $G: X \to X$ by
\begin{equation*}
G(u) = u_0 + \int_0^t \cos(u(s)) - u(s) \, ds
\end{equation*}
Then we see that,
\begin{align*}
  \norm{G(u) - G(v)}_{L^\infty} &= \sup_{0 \leq t \leq T} \left|\int_0^t [\cos(u(s)) - \cos(v(s))] + [u(s) - v(s)] \, ds\right| \\
  &\leq \sup_{0 \leq t \leq T} \left| \int_0^t [\cos(u(s)) - \cos(v(s))] \, ds \right| + \sup_{0 \leq t \leq T} \left| \int_0^t [u(s) - v(s)] \, ds \right| \\
  &\leq \sup_{0 \leq t \leq T} \int_0^t \left| \cos(u(s)) - \cos(v(s)) \right| \, ds + \sup_{0 \leq t \leq T} \int_0^t |u(s) - v(s)| \, ds \\ 
  &\leq \sup_{0 \leq t \leq T} \int_0^t |u(s) - v(s)| \, ds + \sup_{0 \leq t \leq T} \int_0^t |u(s) - v(s)| \, ds \\
  &\leq 2T \norm{u - v}_{L^\infty}
\end{align*}
so then by taking $T < 1/2$ we have that by the Contraction Mapping Theorem that $G$ has a unique fixed point $u$. We can iterate this process to extend the solution uniquely to any $T > 0$.
\subsection*{Part b}

\subsection*{Problem 9.9}
Suppose we have the following differential equation
\begin{equation*}
  \begin{cases}
    -u_{xx} + u - \epsilon u^2 = f(x) \quad \text{for $x \in (0, +\infty)$} \\
    u(0) = u(+\infty) = 0  
  \end{cases}
\end{equation*}
Let $\mathcal{L}: C^2((0,\infty)) \to C^2((0,\infty))$ be the operator defined by
\begin{equation*}
  \mathcal{L}(u) = -u_{xx} + u
\end{equation*}
Then we have that there exists a Green's Function $g$ such that
\begin{equation*}
  G(u) = u(x) = \int_0^\infty g(x,y) \left[f(y) + \epsilon u(y)^2\right] \, dy
\end{equation*}
Then we have that 
\begin{align*}
  \norm{G(u) - G(v)}_{L^\infty} &= \sup_{0 \leq x < \infty} \left| \int_0^\infty g(x,y) \left[ f(y) + \epsilon u(y)^2 - f(y) - \epsilon v(y)^2 \right] \, dy \right| \\
  &\leq \sup_{0 \leq x < \infty} \int_0^\infty |g(x,y)| \left| u(y)^2 - v(y)^2 \right| \, dy \\
  &\leq \sup_{0 \leq x < \infty} \int_0^\infty |g(x,y)| \left| u(y) - v(y) \right| \left| u(y) + v(y) \right| \, dy \\
  &\leq \epsilon \norm{u + v}_{L^\infty} \norm{u-v}_{L^\infty} \sup_{0 \leq x < \infty} \int_0^\infty |g(x,y)| \, dy \\
  &\leq \epsilon 2RM \norm{u-v}_{L^\infty}
\end{align*} 
Then we see that for $\epsilon < \frac{1}{2RM}$ we have that by the Contraction Mapping Theorem that $G$ has a unique fixed point $u$.

\section*{Problem 9.10}
Suppose we have the following differential equation
\begin{equation*}
  \begin{cases}
    \frac{\partial u}{\partial t} - \frac{\partial^3 u}{\partial t \partial x^2} - \epsilon u^3 = f, \quad -\infty < x < \infty, t > 0\\
    u(x,0) = g(x)
  \end{cases}
\end{equation*}
Note that we can rewrite the above as
\begin{equation*}
(1 - \partial_{x}^2)u_t = f + \epsilon u^3 = h
\end{equation*}
then by taking the Fourier Transform we have that
\begin{equation*}
  (1 + \xi^2) \hat{u}_t = \hat{h} 
\end{equation*}
and then we see that it can be formally deduced that  
\begin{equation*}
  u_t = \Tilde{\kappa} * h = \Tilde{\kappa} * (f + \epsilon u^3)
\end{equation*}
where 
\begin{equation*}
  \Tilde{\kappa} = \sqrt{2\pi} \mathcal{F}^{-1}\left(\frac{1}{1+\xi^2}\right) = \frac{1}{2}e^{-|x|}
\end{equation*}
Now by letting $k = -\Tilde{k}_x \in L^1(\R)$ we have that 
\begin{equation*}
  u_t(x,t) = \kappa * (f + \epsilon u^3)  
\end{equation*}
Now by using the Fundemental Theorem of Calculus we get that
\begin{equation*}
  G(u) = u(x,t) = g(x) + \int_0^t \kappa * (f + \epsilon u^3) \, dt
\end{equation*}
To show that $G$ is a contraction map we see that
\begin{align*}
  \norm{G(u) - G(v)}_{L^\infty} &= \sup_{(x,t) \in \R \times [0,T]} \left|\int_0^t \int \kappa * (\epsilon u^3 + \epsilon v^3) \, d\right| 
\end{align*}
\section*{Problem 9.12}
\subsection*{Part a}
We see that $H: X \times \R \to Y$ defined by $H(x,\epsilon) = F(x) + \epsilon G(x)$ is $C^1$ in a neighborhood around $(x_0,0)$ since $DH(x_0,0) = DF(x_0) = 0$. Then we have by the Implicit Function Theorem that there exists a 
unique mapping $g \in C^1$ such that $\epsilon = g(x,y)$. This means the we have that $H(x_0, g(x_0,0)) = 0$ 
\end{document}