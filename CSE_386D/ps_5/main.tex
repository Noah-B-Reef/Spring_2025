\documentclass[12pt]{report}
\usepackage{scribe,graphicx,graphics}
\usepackage{bbm}
\usepackage{yhmath}
\newcommand{\norm}[1]{\left|\left|#1\right|\right|}
\newcommand{\inner}[2]{\left\langle#1,#2\right\rangle}

\course{CSE 386D} 	
\coursetitle{Methods of Applied Mathematics II}	
\semester{Spring 2025}
\lecturer{} % Due Date: {\bf Mon, Oct 3 2016}}
\lecturetitle{Problem Set}
\lecturenumber{4}   
\lecturedate{}    
\usepackage{enumerate}
\newcommand{\remind}[1]{\textcolor{red}{\textbf{#1}}} %To remind me of unfinished work to fix later
\newcommand{\hide}[1]{} %To hide large blocks of code without using % symbols

\newcommand{\ep}{\varepsilon}
\newcommand{\vp}{\varphi}
\newcommand{\lam}{\lambda}
\newcommand{\Lam}{\Lambda}
%\newcommand{\abs}[1]{\ensuremath{\left\lvert#1\right\rvert}} % This clashes with the physics package
%\newcommand{\norm}[1]{\ensuremath{\left\lVert#1\right\rVert}} % This clashes with the physics package
\newcommand{\floor}[1]{\ensuremath{\left\lfloor#1\right\rfloor}}
\newcommand{\ceil}[1]{\ensuremath{\left\lceil#1\right\rceil}}
\newcommand{\A}{\mathbb{A}}
\newcommand{\B}{\mathbb{B}}
\newcommand{\C}{\mathbb{C}}
\newcommand{\D}{\mathbb{D}}
\newcommand{\E}{\mathbb{E}}
\newcommand{\F}{\mathbb{F}}
\newcommand{\K}{\mathbb{K}}
\newcommand{\N}{\mathbb{N}}
\newcommand{\Q}{\mathbb{Q}}
\newcommand{\R}{\mathbb{R}}
\newcommand{\T}{\mathbb{T}}
\newcommand{\X}{\mathbb{X}}
\newcommand{\Y}{\mathbb{Y}}
\newcommand{\Z}{\mathbb{Z}}
\newcommand{\As}{\mathcal{A}}
\newcommand{\Bs}{\mathcal{B}}
\newcommand{\Cs}{\mathcal{C}}
\newcommand{\Ds}{\mathcal{D}}
\newcommand{\Es}{\mathcal{E}}
\newcommand{\Fs}{\mathcal{F}}
\newcommand{\Gs}{\mathcal{G}}
\newcommand{\Hs}{\mathcal{H}}
\newcommand{\Is}{\mathcal{I}}
\newcommand{\Js}{\mathcal{J}}
\newcommand{\Ks}{\mathcal{K}}
\newcommand{\Ls}{\mathcal{L}}
\newcommand{\Ms}{\mathcal{M}}
\newcommand{\Ns}{\mathcal{N}}
\newcommand{\Os}{\mathcal{O}}
\newcommand{\Ps}{\mathcal{P}}
\newcommand{\Qs}{\mathcal{Q}}
\newcommand{\Rs}{\mathcal{R}}
\newcommand{\Ss}{\mathcal{S}}
\newcommand{\Ts}{\mathcal{T}}
\newcommand{\Us}{\mathcal{U}}
\newcommand{\Vs}{\mathcal{V}}
\newcommand{\Ws}{\mathcal{W}}
\newcommand{\Xs}{\mathcal{X}}
\newcommand{\Ys}{\mathcal{Y}}
\newcommand{\Zs}{\mathcal{Z}}
\newcommand{\ab}{\textbf{a}}
\newcommand{\bb}{\textbf{b}}
\newcommand{\cb}{\textbf{c}}
\newcommand{\db}{\textbf{d}}
\newcommand{\ub}{\textbf{u}}
\newcommand{\sbb}{\textbf{s}}
%\renewcommand{\vb}{\textbf{v}} % This clashes with the physics package (the physics package already defines the \vb command)
\newcommand{\wb}{\textbf{w}}
\newcommand{\xb}{\textbf{x}}
\newcommand{\yb}{\textbf{y}}
\newcommand{\zb}{\textbf{z}}
\newcommand{\vbb}{\textbf{v}}
\newcommand{\Ab}{\textbf{A}}
\newcommand{\Bb}{\textbf{B}}
\newcommand{\Cb}{\textbf{C}}
\newcommand{\Db}{\textbf{D}}
\newcommand{\eb}{\textbf{e}}
\newcommand{\ex}{\textbf{e}_x}
\newcommand{\ey}{\textbf{e}_y}
\newcommand{\ez}{\textbf{e}_z}
\newcommand{\zerob}{\mathbf{0}}
\newcommand{\abar}{\overline{a}}
\newcommand{\bbar}{\overline{b}}
\newcommand{\cbar}{\overline{c}}
\newcommand{\dbar}{\overline{d}}
\newcommand{\ubar}{\overline{u}}
\newcommand{\vbar}{\overline{v}}
\newcommand{\wbar}{\overline{w}}
\newcommand{\xbar}{\overline{x}}
\newcommand{\ybar}{\overline{y}}
\newcommand{\zbar}{\overline{z}}
\newcommand{\Abar}{\overline{A}}
\newcommand{\Bbar}{\overline{B}}
\newcommand{\Cbar}{\overline{C}}
\newcommand{\Dbar}{\overline{D}}
\newcommand{\Ubar}{\overline{U}}
\newcommand{\Vbar}{\overline{V}}
\newcommand{\Wbar}{\overline{W}}
\newcommand{\Xbar}{\overline{X}}
\newcommand{\Ybar}{\overline{Y}}
\newcommand{\Zbar}{\overline{Z}}
\newcommand{\Aint}{A^\circ}
\newcommand{\Bint}{B^\circ}
\newcommand{\limk}{\lim_{k\to\infty}}
\newcommand{\limm}{\lim_{m\to\infty}}
\newcommand{\limn}{\lim_{n\to\infty}}
\newcommand{\limx}[1][a]{\lim_{x\to#1}}
\newcommand{\liminfm}{\liminf_{m\to\infty}}
\newcommand{\limsupm}{\limsup_{m\to\infty}}
\newcommand{\liminfn}{\liminf_{n\to\infty}}
\newcommand{\limsupn}{\limsup_{n\to\infty}}
\newcommand{\sumkn}{\sum_{k=1}^n}
\newcommand{\sumk}[1][1]{\sum_{k=#1}^\infty}
\newcommand{\summ}[1][1]{\sum_{m=#1}^\infty}
\newcommand{\sumn}[1][1]{\sum_{n=#1}^\infty}
\newcommand{\emp}{\varnothing}
\newcommand{\exc}{\backslash}
\newcommand{\sub}{\subseteq}
\newcommand{\sups}{\supseteq}
\newcommand{\capp}{\bigcap}
\newcommand{\cupp}{\bigcup}
\newcommand{\kupp}{\bigsqcup}
\newcommand{\cappkn}{\bigcap_{k=1}^n}
\newcommand{\cuppkn}{\bigcup_{k=1}^n}
\newcommand{\kuppkn}{\bigsqcup_{k=1}^n}
\newcommand{\cappk}[1][1]{\bigcap_{k=#1}^\infty}
\newcommand{\cuppk}[1][1]{\bigcup_{k=#1}^\infty}
\newcommand{\cappm}[1][1]{\bigcap_{m=#1}^\infty}
\newcommand{\cuppm}[1][1]{\bigcup_{m=#1}^\infty}
\newcommand{\cappn}[1][1]{\bigcap_{n=#1}^\infty}
\newcommand{\cuppn}[1][1]{\bigcup_{n=#1}^\infty}
\newcommand{\kuppk}[1][1]{\bigsqcup_{k=#1}^\infty}
\newcommand{\kuppm}[1][1]{\bigsqcup_{m=#1}^\infty}
\newcommand{\kuppn}[1][1]{\bigsqcup_{n=#1}^\infty}
\newcommand{\cappa}{\bigcap_{\alpha\in I}}
\newcommand{\cuppa}{\bigcup_{\alpha\in I}}
\newcommand{\kuppa}{\bigsqcup_{\alpha\in I}}
\newcommand{\Rx}{\overline{\mathbb{R}}}
\newcommand{\dx}{\,dx}
\newcommand{\dy}{\,dy}
\newcommand{\dt}{\,dt}
\newcommand{\dax}{\,d\alpha(x)}
\newcommand{\dbx}{\,d\beta(x)}
\DeclareMathOperator{\glb}{\text{glb}}
\DeclareMathOperator{\lub}{\text{lub}}
\newcommand{\xh}{\widehat{x}}
\newcommand{\yh}{\widehat{y}}
\newcommand{\zh}{\widehat{z}}
\newcommand{\<}{\langle}
\renewcommand{\>}{\rangle}
\renewcommand{\iff}{\Leftrightarrow}
\DeclareMathOperator{\im}{\text{im}}
\let\spn\relax\let\Re\relax\let\Im\relax
\DeclareMathOperator{\spn}{\text{span}}
\DeclareMathOperator{\sym}{\text{Sym}}
\DeclareMathOperator{\myskew}{\text{Skew}}
\DeclareMathOperator{\Re}{\text{Re}}
\DeclareMathOperator{\Im}{\text{Im}}
\DeclareMathOperator{\diag}{\text{diag}}
\endinput
\newcommand{\supp}[]{\text{supp}}

% Insert yousupp}
\scribe{Student Name: Noah Reef}

\begin{document}
\maketitle
\section*{Problem 7.1}
Suppose that $f \in H^1(\R^d)$. Then we see that
\begin{equation*}
  \norm{f}_{H^1} = \norm{f}_{W^{1,2}} = \left\{ \norm{f}^2_{L^2} + \norm{Df}^2_{L^2}\right\}^{1/2}
\end{equation*}
recall that 
\begin{equation*}
  \norm{f}_{L^2} = \norm{\hat{f}}_{L^2} 
\end{equation*}
and
\begin{equation*}
  \norm{Df}_{L^2} = \norm{\widehat{Df}}_{L^2} = \norm{i\xi \hat{f}}_{L^2} = |\xi| \norm{\hat{f}}_{L^2}
\end{equation*}
Thus we see that
\begin{equation*}
 \left\{ \norm{f}^2_{L^2} + \norm{Df}^2_{L^2}\right\}^{1/2}= \left\{ \norm{\hat{f}}^2_{L^2} + |\xi|^2 \norm{\hat{f}}^2_{L^2}\right\}^{1/2} = \int_{\R^d} (1 + |\xi|^2) |\hat{f}(\xi)|^2 d\xi 
\end{equation*}
and in general for $f \in H^k(\R^d)$ we have that
\begin{equation*}
  \norm{f}^2_{H^k} = \left\{ \norm{f}^2_{L^2} + \sum_{j=1}^{k}\norm{D^jf}^2_{L^2}\right\}^{1/2} = \int_{\R^d} \sum_{|\alpha| \leq k} |\xi^{\alpha}|^2 |\hat{f}(\xi)|^2 d\xi = \int_{\R^d} (1 + |\xi|^2)^k |\hat{f}(\xi)|^2 d\xi
\end{equation*}

\section*{Problem 7.3}
Let $f \in H^1(\R^d)$ and we define 
\begin{equation*}
  \delta_0(f) = f(0)
\end{equation*}
note that for $d = 1$ we have that $H^1(\R)$ is continously embedded in $C^0(\R)$ and hence there exists a $C$ such that
\begin{equation*}
  \norm{f}_{C^0} \leq C \norm{f}_{H^1}
\end{equation*}
and from the fundemental theorem of calculus, we get that
\begin{equation*}
f(x) - f(0) = \int_0^x f'(t) dt
\end{equation*}
which then, by using Cauchy-Schwartz, gives us 
\begin{equation*}
  |f(0)| \leq \norm{f}_{C^0} \leq C \norm{f}_{H^1}
\end{equation*}
that is $\delta_0$ is a bounded linear functional on $H^1(\R)$ and hence in $(H^1(\R))^*$. However for $d \geq 2$ we consider the sequence $f_n(x) = \phi(nx)$ where $\phi \in C_0^\infty(\R^d)$ and $\phi(0) = 1$. Then we see that 
\begin{align*}
  \norm{f_n}^2_{H^1} = \norm{f_n}_{W^{1,2}} = \left\{ \norm{f_n}^2_{L^2} + \norm{Df_n}^2_{L^2}\right\} &= \frac{1}{n^d} \norm{\phi}^2_{L^2} + \frac{n^2}{n^d} \norm{\nabla \phi}^2_{L^2} \leq C n^{1 - \frac{d}{2}}\\ 
\end{align*}
then we see that as $n \to \infty$ we have that $\norm{f_n}_{H^1} \to 0$ but $\delta_0(f_n) = 1$ and hence $\delta_0$ is not a bounded linear functional on $H^1(\R^d)$ for $d \geq 2$.

\section*{Problem 7.7}
Let $u \in \mathcal{D}'(\R^d)$ and $\phi \in \mathcal{D}(\R^d)$. 

\subsection*{Part a}
Note that by the fundamental theorem of calculus we have that
\begin{equation*}
  u(\tau_{y}\phi) - u(\tau_0 \phi) = \int_0^1 \frac{d}{dt} u(\tau_{ty}\phi) dt 
\end{equation*}
and we see that by applying the usual chain rule we get
\begin{equation*}
  \frac{d}{dt} \tau_{ty}\phi = \sum_{j=1}^d y_j \partial_j \phi(\tau_{ty}x) 
\end{equation*}
and hence we see that
\begin{equation*}
  \frac{d}{dt} u(\tau_{ty}\phi) = \sum_{j=1}^d y_j \partial_j u(\tau_{ty}\phi)
\end{equation*}
and finally we get that
\begin{equation*}
  u(\tau_{y}\phi) - u(\tau_0 \phi) = \int_0^1 \sum_{j=1}^d y_j \partial_j u(\tau_{ty}\phi) dt = \sum_{j=1}^d y_j \int_0^1 \partial_j u(\tau_{ty}\phi) dt
\end{equation*}

\subsection*{Part b}
Let $f \in W_{\text{loc}}^{1,1}(\R^d)$ and hence $f \in L^1_{\text{loc}}(\R^d)$ and hence $f \in \mathcal{D}'(\R^d)$. Then we see that
\begin{equation*}
  \inner{\tau_{-y}f}{\phi} - \inner{f}{\phi} =\sum_{j=1}^d y_j \int_0^1 \partial_j f(\tau_{ty}\phi) dt
\end{equation*} 
which is equivalent to
\begin{equation*}
  \sum_{j=1}^d y_j \int_0^1 \int_{\R^d} \partial_j f(\tau_{ty}x) \phi(x) dx dt = \int_0^1  y \cdot \nabla f(x + ty) \,dt \phi(x) dx
\end{equation*}
which implies that
\begin{equation*}
  f(x + y) - f(x) = \int_0^1 y \cdot \nabla f(x + ty) \, dt 
\end{equation*}

\subsection*{Part c}
From b we have that
\begin{equation*}
  |f(x+y) - f(x)| \left|\int_0^1 y \cdot \nabla f(x+ty) \, dt\right| \leq |y| \int_0^1 |\nabla f(x+ty)| \, dt
\end{equation*}
then for any fixed ball $B_R(0)$ if $f \in W^{1,1}_{\text{loc}}(\R^d)$ then by taking $L_{R,f} = \sup_{x \in B_R(0)} |\nabla f(x)|$ we see that
\begin{equation*}
  |f(x+y) - f(x)| \leq |y| L_{R,f}
\end{equation*}
and hence $f$ is locally Lipschitz continuous and hence $W^{1,1}_{\text{loc}}(\R^d) \subseteq C^{0,1}_{\text{loc}}(\R^d)$.



\section*{Problem 7.8}
\subsection*{Part a}
Let $\Omega \subseteq \R^d$ be bounded and contains $0$. Let $f(x) = |x|^\alpha$ and $1 \leq p < d$ and $q > dp/(d-p). Then if we consider
\begin{equation*}
  1-\frac{d}{p} < \alpha \leq -\frac{d}{q} 
\end{equation*}
since we have that for $f \in W^{1,p}(\Omega)$ we have that
\begin{equation*}
  \norm{f}_{W^{1,p}} = \norm{f}^p_{L^p} + \norm{Df}^p_{L^p} = \int_0^R r^{\alpha p} r^{d-1} dr + \alpha \int_0^R r^{(\alpha-1) p} r^{d-1} dr
\end{equation*}
which we see that both integrals converge for $\alpha > 1 - (d/p)$ and hence $f \in W^{1,p}(\Omega)$. Additionally we have that 
\begin{equation*}
  \norm{f}_{L^q(\Omega)}^q = \int_0^R r^{q\alpha} r^{d-1} dr 
\end{equation*}
which only converges for $q\alpha + d > 0$, however since $\alpha \leq -d/q$ we see that $q\alpha + d \leq 0$ and hence $f \notin L^q(\Omega)$.

\subsection*{Part b}
Since we see that our function $f$ in part a is such that $f \not\in L^q(\Omega)$ for $\alpha \leq -d/q$
and hence the Dirac mass is in $W^{-s,p}(\Omega)$ for $s > d/p$ and $1 \leq p < d$.

\subsection*{Part c}
Let $\Omega \subseteq \R^d = B_R(0)$ and let $f(x) = \log(\log(4R/|x|))$. Then we see that
\begin{equation*}
  \norm{f}_{W^{1,p}_{}(B_R(0))}^p = \norm{f}^p_L^p_{(B_R(0))} + \norm{Df}^p_L^p_{(B_R(0))}
\end{equation*}
Note that 
\begin{align*}
  \norm{f}^p_L^p(B_R(0)) &= \int_{B_R(0)} \left|\log(\log(4R/|x|))\right|^d \dx = \int_0^R \left|\log(\log(4R/r))\right|^d r^{d-1} dr  \\ 
  \norm{Df}^p_L^p(B_R(0)) &= \int_{B_R(0)} \left|\frac{1}{|x| \log(4R/|x|)}\right|^d \dx = \int_0^R \left|\frac{1}{r \log(4R/r)}\right|^d r^{d-1} dr 
\end{align*}
note that
\begin{equation*}
  \int_0^R \left|\log(4R/r)\right|^d r^{d-1} dr \leq \int_0^R r^{2d - 1} \, dr < \infty
\end{equation*}
and similarly
\begin{equation*}
  \int_0^R \left|\frac{1}{r \log(4R/r)}\right|^d r^{d-1} dr = \int_0^R \frac{1}{r |\log(4R/r)|^d} \, dr
\end{equation*}
and making the subsitution $r = 4Re^{-u}$ and $dr = -4Re^{-u} du$ we see that
\begin{equation*}
  \int_0^R \frac{1}{r |\log(4R/r)|^d} \, dr = \int_{u_0}^\infty \frac{1}{4Re^{-u} |u|^d} 4Re^{-u} \, du = \int_{u_0}^\infty \frac{1}{|u|^d} \, du < \infty
\end{equation*}
thus $f \in W^{1,p}(B_R(0))$ for $1 \leq p = d$, but $f \notin L^\infty(\Omega)$.

\subsection*{Part d}
Let $\Omega = (-1,1)$ and $u(x) = |x|$, then we see that
\begin{equation*}
  \norm{u}_{W^{1,\infty}} = \max\{\norm{u}_{L^\infty}, \norm{Du}_{L^\infty}\} 
\end{equation*}
where
\begin{equation*}
  \norm{u}_{L^\infty} = \max_{x \in \Omega} |u(x)| = 1
\end{equation*}
and
\begin{equation*}
  \norm{Du}_{L^\infty} = \max_{x \in \Omega} |Du(x)| = 1
\end{equation*}
and hence
\begin{equation*}
  \norm{u}_{W^{1,\infty}} = 1
\end{equation*}
and thus $u \in W^{1,\infty}(\Omega)$. However we see that if $\{u_k\}_{k=1}^\infty \subseteq C^\infty$ are such that $u_k \to u$ in $C^\infty(\Omega)$, but we see that
\begin{equation*}
  \norm{u_k - u}_{W^{1,\infty}} = \max\{\norm{u_k - u}_{L^\infty}, \norm{Du_k - Du}_{L^\infty}\} 
\end{equation*}
and we see that
\begin{equation*}
  \norm{u_k - u}_{L^\infty} = \max_{x \in \Omega} |u_k(x) - u(x)| = 0
\end{equation*}
however since $u_k \in C^\infty$ we see that $\lim_{x \to 0^+} Du_k(x) = 1$ and $\lim_{x \to 0^-} Du_k(x) = -1$ and hence there must exist an $N$ such that for all $k \geq N$ we have that 
\begin{equation*}
  \lim_{x \to 0^+} |Du_k(x) - 1| < \epsilon \quad \text{and} \quad \lim_{x \to 0^-} |Du_k(x) + 1| < \epsilon
\end{equation*}
however this is a contradiction since $Du(x)$ is not continuous at $x = 0$ and hence $u_k$ does not converge to $u$ in $W^{1,\infty}(\Omega)$.

\end{document}
