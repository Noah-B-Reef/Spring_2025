\documentclass{article}

\usepackage{amsmath, amsthm, amssymb, amsfonts}
\usepackage{thmtools}
\usepackage{graphicx}
\usepackage{setspace}
\usepackage{geometry}
\usepackage{float}
\usepackage{hyperref}
\usepackage[utf8]{inputenc}
\usepackage[english]{babel}
\usepackage{framed}
\usepackage[dvipsnames]{xcolor}
\usepackage{tcolorbox}
\usepackage{proba}

\colorlet{LightGray}{White!90!Periwinkle}
\colorlet{LightOrange}{Orange!15}
\colorlet{LightGreen}{Green!15}

\newcommand{\HRule}[1]{\rule{\linewidth}{#1}}

\declaretheoremstyle[name=Theorem,]{thmsty}
\declaretheorem[style=thmsty,numberwithin=section]{theorem}
\tcolorboxenvironment{theorem}{colback=LightGray}

\declaretheoremstyle[name=Proposition,]{prosty}
\declaretheorem[style=prosty,numberlike=theorem]{proposition}
\tcolorboxenvironment{proposition}{colback=LightOrange}

\declaretheoremstyle[name=Principle,]{prcpsty}
\declaretheorem[style=prcpsty,numberlike=theorem]{principle}
\tcolorboxenvironment{principle}{colback=LightGreen}

\setstretch{1.2}
\geometry{
    textheight=9in,
    textwidth=5.5in,
    top=1in,
    headheight=12pt,
    headsep=25pt,
    footskip=30pt
}

% ------------------------------------------------------------------------------

\begin{document}

% ------------------------------------------------------------------------------
% Cover Page and ToC
% ------------------------------------------------------------------------------

\title{ \normalsize \textsc{}
		\\ [2.0cm]
		\HRule{1.5pt} \\
		\LARGE \textbf{\uppercase{Foundational Techniques in Machine learning \& Data Science}
		\HRule{2.0pt} \\ [0.6cm] \LARGE{CSE 382M} \vspace*{10\baselineskip}}
		}
\date{}
\author{\textbf{Author} \\ 
		Noah Reef \\
		UT Austin \\
		Spring 2025}

\maketitle
\newpage

\tableofcontents
\newpage

% ------------------------------------------------------------------------------
\section{Probability}
\subsection{Concentration Inequalities}
\begin{theorem}[Markov's Inequality]
  Let $x$ be a non-negative random variable. Then for $a > 0$,
  \begin{equation*}
    \prob{(x \geq a)}\leq \frac{\EX{x}}{a}
  \end{equation*}
\end{theorem}

\begin{theorem}[Chebyshev's Inequality]
  Let $x$ be a random variable. Then for $c > 0$,
  \begin{equation*}
    \prob{(|x - \EX{x}| \geq c)} \leq \frac{\VarX{x}}{c^2}
  \end{equation*}
\end{theorem}

\begin{theorem}[Law of Large Numbers]
  Let $x_1,x_2,\dots, x_n$ be $n$ independent samples of a random variable $x$. Then
  \begin{equation*}
    \probX{\left|\frac{x_1 + x_2 + \dots + x_n}{n} - \EX{x}\right| \geq \epsilon} \leq \frac{\VarX{x}}{n\epsilon^2}
  \end{equation*}
\end{theorem}
\begin{theorem}[Master Tail Bounds Theorem]
  Let $x = x_1 + x_2 + \dots + x_n$, where $x_1,x_2,\dots,x_n$ are mutually independent random variables with zero mean and variance at most $\sigma^2$. Let $0 \leq a \leq \sqrt{2}n\sigma^2$. Assume that $|\EX{x_i^s}| \leq \sigma^2 s!$ for $s = 3,4,\dots, \lfloor(a^2/4n\sigma^2)\rfloor$. Then,
  \begin{equation*}
    \probX{|x| \geq a} \leq 3e^{-a^2/(12n\sigma^2)}
  \end{equation*}
\end{theorem}

\begin{theorem}[General Master Tail Bounds Theorem]
  Let $x = x_1 + x_2 + \dots + x_n$, where $x_1,x_2,\dots,x_n$ are mutually independent random variables with zero mean and variance at most $\sigma^2$. Let $0 \leq a \leq \sqrt{2}n\sigma^2$ and $s \leq n\sigma^2/2$ is a positive even integer and $|\EX{x_i^r} \leq \sigma^2 r!$ for $r = 3,4,\dots,s$.Then,
  \begin{equation*}
    \probX{|x_1 + x_2 + \dots + x_n| \geq a} \leq \left(\frac{2sn\sigma^2}{a^2}\right)^{s/2}
  \end{equation*}
  If further, $s \geq a^2/(4n\sigma^2)$, then we also have:
  \begin{equation*}
    \probX{|x_1 + x_2 + \dots + x_n| \geq a} \leq 3e^{-a^2/(12n\sigma^2)}
  \end{equation*}
\end{theorem}

\begin{theorem}[Chernoff Bound]
  Let $x$ be a Bernoulli random variable, with $\EX{x} = p$ and $\VarX{x} = p(1-p)$ then we have that
  \begin{equation*}
    \probX{\left|\frac{y}{n} - p \right| \geq \sqrt{2}c p(1-p)} \leq 3 e^{-np(1-p)c^2/6}
  \end{equation*}
\end{theorem}


% ------------------------------------------------------------------------------
% Reference and Cited Works
% ------------------------------------------------------------------------------

\bibliographystyle{IEEEtran}
\bibliography{References.bib}

% ------------------------------------------------------------------------------

\end{document}
